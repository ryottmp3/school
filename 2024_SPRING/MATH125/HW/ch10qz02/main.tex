\documentclass[12pt, letterpaper]{article}
\usepackage[includehead]{geometry}
\usepackage{fancyhdr}
\usepackage[british]{babel}
\usepackage{mathtools}
\usepackage{mathrsfs}
\usepackage{amsmath}
\usepackage{amsthm}
\usepackage{amssymb}
\usepackage{amstext}
\usepackage{soul}
\usepackage{color}
\usepackage{colortbl}
\usepackage{siunitx}
\usepackage{textgreek}
\usepackage{xspace}
\usepackage{lastpage}
\usepackage{lipsum}
\definecolor{error}{rgb}{255,255,0}
\newcommand{\degree}{\ensuremath{{}^{\circ}}\xspace}


\fancypagestyle{plain}{%
\fancyhf{}%
\fancyhead[L]{MATH 125---M03 \\ Dr\@. May}
\fancyhead[R]{H. Ryott Glayzer \\ \today}%
\fancyhead[C]{Chapter 10 Quiz 2 \\\;}%
\fancyfoot[L]{\textsc{Ch 10 Quiz 2}}%
\fancyfoot[C]{\thepage~of~\pageref{LastPage}}
\fancyfoot[R]{\textsc{Calc II with Dr\@. May}}
}

%%%%%%%%%%%%%%%%%%%%%%%%%%%%%%%%%%%%%%%%%%%%%%%%%%%%%%%%%%%%%%%%%%%%%%%%%%%%%%%
\begin{document}



\title{Calculus II\@: Chapter 10 Quiz 2}
\author{H. Ryott Glayzer}
\date{\today}


\maketitle


%%%%%%%%%%%%%%%%%%%%%%%%%%%%%%%%%%%%%%%%%%%%%%%%%%%%%%%%%%%%%%%%%%%%%%%%%%%%%%%
\section{Find the Taylor Series for $f(x) = \cos{(x)}$ centered at $a=0$.
	For what $x$-values does the series converge?}

The Taylor series for $f(x)$, centered at $a$ can be defined as: 

\begin{equation}
	\sum_{n=0}^{\infty} \frac{f^{(n)}(a)}{n!} {(x-a)}^{n}\,.	
\end{equation}

Taking $f(x)$ to be defined as $\cos{(x)}$ and $a = 0$ gives us the first few terms

\begin{equation}
	\frac{\cos{(0)}{{(x-0)}^{0}}}{0!} +
	\frac{-\sin{(0)}{(x-0)}^{1}}{1!} +
	\frac{-\cos{(0)}{(x-0)}^{2}}{2!} +
	\frac{\sin{(0)}{(x-0)}^{3}}{3!} + \cdots \,,
\end{equation}
which can be simplified to
\begin{equation}
	1 - \frac{x^{2}}{2!} + \frac{x^{4}}{4!} + \cdots \,.
\end{equation}
This can be generalized in series notation as the power series
\begin{equation}
	\sum_{n=0}^{\infty} {(-1)}^{n}\frac{x^{2n}}{(2n)!} \,.
\end{equation}

To determine the interval of convergence for this series, we must use
a convergence test.
Since a factorial is involved, we will use the ratio test for convergence.

We define $a_n$ as $\frac{x^{2n}}{{(2n)}!}$\,, and $a_{n+1}$ as
$\frac{x^{2{(n+1)}}}{{(2{(n+1)})}!}$\,.
Thus, by the definition of the ratio test, we find
\begin{subequations}
\begin{align}
	\textup{L} &=
	\lim_{n\to\infty}
	\left|
	\frac
	{\frac{{(-1)}^{n}x^{2n+2}}{{(2n+2)}!}}
	{\frac{{(-1)}^{n}x^{2n}}{{(2n)}!}}
	\right|\\
	  &= \lim_{n\to\infty}
	\left|
	\frac
	{x^{2n+2}\cdot{(2n)}!}
	{x^{2n}\cdot{(2n+2)}!}
	\right|\\
	  &=
	   \lim_{n\to\infty}
	   \left|
	   \frac
	   {x^{2}}
	   {{(2n+1)}{(2n+2)}}
	   \right|\\
	  &=
	  \lim_{n\to\infty}
	  \left|
	  \frac{x^2}{2{(n+1)}{(2n+1)}}
	  \right|
\end{align}
\end{subequations}
In this limit, the $x$-values, while variable,
do not necessarily increase with $n$ and can be treated as an arbitrary
constant of sorts.
\begin{subequations}
\begin{align}
	L &= 
	\lim_{n\to\infty}
	\left|
	\frac{x^2}{4n^2+6n+2}
	\right|\to0\\
	  &= 0
\end{align}
\end{subequations}
Because of this, it follows that $a_n$ converges absolutely,
for all values of $x$.
In other words, $x$ converges on the interval ${(-\infty,\infty)}$.

\hfill\blacksquare%
\clearpage


%%%%%%%%%%%%%%%%%%%%%%%%%%%%%%%%%%%%%%%%%%%%%%%%%%%%%%%%%%%%%%%%%%%%%%%%%%%%%%%
\section{Give the 4th-order Taylor Polynomial, $p_{4}{(x)}$, for 
$f{(x)}=\cos{(x)}$, centered at $a=0$}

The Taylor Polynomial can be derived from the Taylor Series as such:
\begin{equation}
	p_{k}{(x)} = \sum_{n=0}^{k} \frac{f^{(n)}{(a)}}{n!}{(x-a)}^{n} \,.
\end{equation}
This allows us to find the fourth Taylor polynomial for the provided function
$f{(x)}=\cos{(x)}$:
\begin{subequations}
	\begin{align}
		p_{0}{(x)} &= 1\\
		p_{1}{(x)} &= 1 - 0x\\
		p_{2}{(x)} &= 1 - 0x - \frac{1}{2}x^{2}\\
		p_{3}{(x)} &= 1 - 0x - \frac{1}{2}x^{2} - 0x^{3}\\
		p_{4}{(x)} &= 1 - 0x - \frac{1}{2}x^{2} - 0x^{3} + \frac{1}{24}x^{4}\\
		\leadsto p_{4}{(x)} &= 1 - \frac{1}{2}x^{2} + \frac{1}{24}x^{4}\,.
	\end{align}
\end{subequations}
\hfill\blacksquare%

%%%%%%%%%%%%%%%%%%%%%%%%%%%%%%%%%%%%%%%%%%%%%%%%%%%%%%%%%%%%%%%%%%%%%%%%%%%%%%%
\subsection{Provide the maximum error, $|R_{4}{(x)}|=|\cos{(x)}-p_{4}{(x)}|$,
for $p_{4}{(x)}$ over $-\frac{\pi}{2}\leq x \leq \frac{\pi}{2}$.}
To find maximum error, we must use Lagrange's formula:
\begin{equation}
	R_{n}{(x)} = \frac{f^{(n+1)}{(c)}}{(n+1)!}{(x-a)}^{n+1}
\end{equation}
This leads us to the equation
\begin{equation}
	R_{4}{(x)}=\frac{\Dot{\Ddot{\Ddot{f}}}{(c)}}{5!}{x^{5}}=\frac{-\sin{c}}{120}x^{5}
\end{equation}
over the interval ${(-\frac{\pi}{2}, \frac{\pi}{2})}$.
Since both $f{(x)}$ and $p_4{(x)}$ are symmetric about the $y$-axis, we can
take the interval to be $0 \leq |c| \leq \frac{\pi}{2}$.
Looking at the trends of $R_4{(x)}$ as we vary $c$ from 0 $-\frac{\pi}{2}$ to
$\frac{\pi}{2}$ allows us to see that the greatest error will occur
when $|c|$ is equal to $\frac{\pi}{2}$. 

Thus, utilizing this error value for the given equation 
\begin{equation}
	|R_4{(x)}| = |\cos{(x)}-p_4{(x)}|
\end{equation}
provides the maximum error over the interval ${(-\frac{\pi}{2}, \frac{\pi}{2})}$
of  $0.01996$


\hfill\blacksquare%
%%%%%%%%%%%%%%%%%%%%%%%%%%%%%%%%%%%%%%%%%%%%%%%%%%%%%%%%%%%%%%%%%%%%%%%%%%%%%%%
\subsection{Compute $p_{4}{(1)}$ and $f{(1)}$ and round to the fourth decimal
place. Compare the difference to the upper bound of $|R_{4}{(1)}|$.}
$p_{4}{(1)}=\frac{13}{24} \sim 0.5417$\\
$f{(1)}\sim 0.5403$\\
$\textup{diff}\sim 0.0014$\\
The upper bound can be given via 
\begin{equation}
	R_{4}{(x)}=\frac{\Dot{\Ddot{\Ddot{f}}}{(c)}}{5!}{x^{5}}=\frac{-\sin{c}}{120}x^{5}\,,
\end{equation}
choosing an appropriate value for $c$ given $x$ is 1.
Upon analyzing this equation for the parametrization of $c$, we can see that
the upper bound occurs when $c=1$.
Plugging into this equation gives us $R_{4}{(1)}=-0.00701$
This is about half of the difference between the two earlier values.

\hfill\blacksquare%
\clearpage
%%%%%%%%%%%%%%%%%%%%%%%%%%%%%%%%%%%%%%%%%%%%%%%%%%%%%%%%%%%%%%%%%%%%%%%%%%%%%%%
\section{Find the Taylor Series for $f{(x)}=e^x$ centered at $a=0$.
For what $x$-values does this series converge?}

The Taylor series for $f(x)$, centered at $a$ can be defined as: 

\begin{equation}
	\sum_{n=0}^{\infty} \frac{f^{(n)}(a)}{n!} {(x-a)}^{n}\,.	
\end{equation}

Taking $f(x)$ to be defined as $e^x$ and $a = 0$ gives us the first few terms

\begin{equation}
	\frac{e^{0}{{(x-0)}^{0}}}{0!} +
	\frac{e^{0}{(x-0)}^{1}}{1!} +
	\frac{e^{0}{(x-0)}^{2}}{2!} +
	\frac{e^{0}{(x-0)}^{3}}{3!} + \cdots \,,
\end{equation}
which can be simplified to the power series
\begin{equation}
	\sum_{n=0}^{\infty} \frac{x^n}{n!}
\end{equation}
which converges over $x \in \mathbb{R} : x > 0$.

\hfill\blacksquare%
%%%%%%%%%%%%%%%%%%%%%%%%%%%%%%%%%%%%%%%%%%%%%%%%%%%%%%%%%%%%%%%%%%%%%%%%%%%%%%%

\begin{center}
	The rest is on paper, as I don't have time to type out pretty \LaTeX%
\end{center}

































\end{document}





%%%%%%%%%%%%%%%%%%%%%%%%%%%%%%%%%%%%%%%%%%%%%%%%%%%%%%%%%%%%%%%%%%%%%%%%%%%%%%%
%%%                                 Copypasta                               %%%
%%%%%%%%%%%%%%%%%%%%%%%%%%%%%%%%%%%%%%%%%%%%%%%%%%%%%%%%%%%%%%%%%%%%%%%%%%%%%%%

% Table for questions with multiple parts

% \begin{center}
% 	\begin{tabular}{|c|c|c|c|c|}
% 		\hline
% 		_ & _ & _ & _ & _ \\
% 		\line
% 		_ & _ & _ & _ & _ \\
% 		\line
% 	\end{tabular}
% \end{center}
