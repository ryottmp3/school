\documentclass[12pt, letterpaper]{article}
%\usepackage[includehead, margin=0.75in]{geometry}
\usepackage{fancyhdr}
\usepackage[british]{babel}


\begin{document}

\fancypagestyle{plain}{%
\fancyhf{}
\fancyhead[L]{CHEM 112L-M60 \\ Dr.\@ Moulder \\TA: Xiaoyu Zhang}
\fancyhead[R]{H. Ryott Glayzer \\\:\\ \today}
\fancyhead[C]{Lab \#8 Pre-Lab Report \\ Moles \\ \:}
}


\title{Lab \#8 Pre-Lab Report \\ \large Moles}
\author{H. Ryott Glayzer}
\date{\today}


\maketitle


\section*{Notice of ADA Accommodation}
I have an ADA accommodation to type out my pre-lab report when my disability flares up.
This document is a utilization of that accommodation.

\section{Question One}
\begin{quote}
    What is the point of this lab? Define the chemical principles we are testing in your own words.
\end{quote}
This lab teaches students to do stoichiometric calculations and enforces the validity of 
the law of conservation of mass and energy, the definite proportion law, and Dalton's atomic
theory.
Students must explore concepts like molar mass and molar ratios of molecules to determine
a molecular empirical formula. 
Students will learn proper cleaning, usage, and storage of crucibles and Bunsen burners.
Students will also explore the hydration of Epsom salt. 
Nile Red did an interesting video on this topic.

\section{Question Two}
\begin{quote}
    What is the logic of this lab? How do the procedures test the hypothesis that the chemical 
    processes are correct?
\end{quote}
This lab encourages students to explore the concepts of stoichiometry we are learning in class.
Hopefully, this will help us pass the test. 
These procedures help students reinforce the basics of stoichiometry by proving the conservation
of mass and energy and the law of definite proportions.
This lab also may engage students who may not be generally engaged in labs, since there is 
fire involved.

\section{Question Three}
\begin{quote}
    Where are potential problem points in the procedures? Where is it easy to make an error 
    or have something just go wrong?
\end{quote}
We are dealing with fire in this lab, and that fact in and of itself is a big red flag for 
things that \textit{could} go wrong.
Students need to follow procedures closely and listen intently to the GTA to ensure that
hazards are mitigated and the experiment is done safely.
Another thing that could go wrong is students not accurately massing their samples, or doing
some bad algebra and getting the empirical formulae wrong.
To mitigate this, students will need to double check their maths and ensure they are 
using massing equipment correctly.

\section{Question Four}
\begin{quote}
    What are the health and safety hazards for this lab and how do we minimize them?
\end{quote}
 We are dealing with magnesium, which burns with a very high luminosity.
 Students who do not follow instructions could cause permanent damage to their corneas.
 To mitigate damage to eyeballs, students need to follow the written and verbal
 instructions closely.
 Dealing with Bunsen burners is also a safety hazard.
 To mitigate risks associated with fire, students should ensure they are following
 GTA instructions and lab safety rules.

\end{document}
