\documentclass[11pt, letterpaper]{article}
\usepackage[includehead, margin=0.75in]{geometry}
\usepackage{fancyhdr}
\usepackage[british]{babel}


\begin{document}

\fancypagestyle{plain}{
\fancyhf{}
\fancyhead[L]{CHEM 112L - M60 \\ Dr. Moulder \\TA: Xiaoyu Zhang}
\fancyhead[R]{H. Ryott Glayzer \\\:\\ 26 Mar 2024}
\fancyhead[C]{Lab \#10 Pre-Lab Report \\ Hess's Law \& Thermochemistry \\ \:}
}


\title{Lab \#10 Pre-Lab Report \\ \large Hess's Law \& Thermochemistry}
\author{H. Ryott Glayzer}
\date{26 Mar 2024}


\maketitle


\section*{Notice of ADA Accommodation}
I have an ADA accommodation to type out my pre-lab report when my disability flares up.
This document is a utilization of that accommodation.

\section{Question One}
\begin{quote}
    What is the point of this lab? Define the chemical principles we are testing in your own words.
\end{quote}
This lab explores the laws of thermochemistry. By measuring enthalpy changes in chemical reactions,
students stand to gain deeper understanding of important concepts in chemistry that will prepare
students not only for success in the lecture portion of this class, but also for other studies.
These thermochemical concepts of specific heat, enthalpy change, and calorimetry ensure that
students are prepared for more advanced concepts that will directly affect their major.

\section{Question Two}
\begin{quote}
    What is the logic of this lab? How do the procedures test the hypothesis that the chemical 
    processes are correct?
\end{quote}
This lab explores a few different concepts in thermochemistry.
The experiments that use a coffee cup calorimeter test calorimetry
by isolating both mass transfer and heat transfer, creating a closed system
and taking measurements inside that closed system.
This system is not perfectly closed, though, and the lab thus tests students' 
abilities to utilize critical thinking in the experiment space to estimate
uncertainties and use significant figures in their data.
Students will utilize the equations provided by the laws of thermochemistry to
make physical and chemical observations on the experiments.
These observations will be used to provide experimental confirmation of important
thermochemical laws and theories.

\section{Question Three}
\begin{quote}
    Where are potential problem points in the procedures? Where is it easy to make an error 
    or have something just go wrong?
\end{quote}
Problems can arise in the experiments with students not following procedures or TA instructions.
Students may also introduce uncertainties and errors in their data by failing to utilize
significant figures appropriately, making algebra mistakes, and making measurements incorrectly.
These problems can all be mitigated by taking extra care to measure multiple times and recheck
any calculations.

\section{Question Four}
\begin{quote}
    What are the health and safety hazards for this lab and how do we minimize them?
\end{quote}
Health and safety hazards arise in this lab with handling \ce{HCl}, an acid that is
dangerous if contact with the human body occurs, except of course in the stomach.
Students must take caution in handling this chemical and ensure that all safety 
precautions and TA instructions are followed.
The base \ce{NaOH} presents a similar danger and must be handled similarly.
In experiment 10.7.3, water is boiled, which presents a burn hazard.
Caution should be taken to mitigate direct contact with the water.
If any injury should occur, students should notify the TA immediately and 
follow appropriate harm mitigation procedures.

\end{document}

