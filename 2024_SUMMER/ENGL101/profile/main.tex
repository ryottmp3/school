\documentclass[12pt, letterpaper]{article}

%
%Margin - 1 inch on all sides
%
\usepackage[letterpaper]{geometry}
\usepackage{times}
\geometry{top=1.0in, bottom=1.0in, left=1.0in, right=1.0in}
\usepackage[T1]{fontenc}
\usepackage[utf8]{inputenc}
\usepackage{lmodern}
%
%Doublespacing
%
\usepackage{setspace}
\doublespacing

%
%Rotating tables (e.g. sideways when too long)
%
\usepackage{rotating}


%
%Fancy-header package to modify header/page numbering (insert last name)
%
\usepackage{fancyhdr}
\pagestyle{fancy}
\lhead{} 
\chead{} 
\rhead{Glayzer \thepage} 
\lfoot{} 
\cfoot{} 
\rfoot{} 
\renewcommand{\headrulewidth}{0pt} 
\renewcommand{\footrulewidth}{0pt} 
%To make sure we actually have header 0.5in away from top edge
%12pt is one-sixth of an inch. Subtract this from 0.5in to get headsep value
\setlength\headsep{0.333in}

%
%Works cited environment
%(to start, use \begin{workscited...}, each entry preceded by \bibent)
% - from Ryan Alcock's MLA style file
%
\newcommand{\bibent}{\noindent \hangindent 40pt}
\newenvironment{workscited}{\newpage \begin{center} Works Cited \end{center}}{\newpage }


%
%Begin document
%
\begin{document}
\begin{flushleft}

%%%%First page name, class, etc
H.\@ Ryott Glayzer\\
Dr. Herrick\\
ENGL 101\\
20 May 2024\\


%%%%Title
\begin{center}
Tony Long Wolf: Wakáŋ Waš'ágya Nážiŋ
\end{center}


%%%%Changes paragraph indentation to 0.5in
\setlength{\parindent}{0.5in}
%%%%Begin body of paper here

%%%%%%%%%%%%%%%%%%%
%   Rough Draft   %
%%%%%%%%%%%%%%%%%%%

An old Lakota saying my uŋčí told me goes, 
``Wakáŋ kiŋ hená slolyápi čhaŋkhé hót\v{h}aŋiŋpi éyaš tuwéni anáwičhaǧoptaŋpi šni.''
Roughly translated to English, this means 
``The elders know these things and speak about them, but no one listens to them.''
In Lakota tradition, elders are respected and sought out for wisdom.
Tony Long Wolf is a Lakota elder who is standing strong in the traditional Lakota Way, or Wólak\v{h}ota.
He works at Oáye Lúta Ok\v{h}ólakičhiye as a cultural advocate, where he translates and interprets
messages from medicine men.
\vspace{5mm}

Tony Long Wolf was born on the twelfth of August in 1951 on the Oglála-Oy\'aŋke (Pine Ridge Indian Reservation).
Growing up, his family had no running water or electricity.
Life on the Reservation was hard.
Alcoholism and abuse were prevalent in all aspects of life.
Tony's father, Antonio Long Wolf, would move around from
town to town working for different \textit{wasíču} (white people).
When he was in the third grade, Tony had to go to the Holy Rosary Mission,
known today as the Red Cloud Indian School.
The Holy Rosary Mission was one of the infamous Indian Boarding Schools.
Tony says that his experience at Holy Rosary forever altered his life.
\vspace{5mm}

Abuse was rampant in the Holy Rosary Boarding school.
Students were ostracized and punished if they ever spoke their language
or showed that they were proud to be Lakota.
Students who didn't make it on the honor roll were punished severely.
The physical and emotional abuse that went on at this school and
others like it have recently made them infamous.
However, Tony says that the sexual abuse is what scarred him the most.
% Tony went very in-depth but I won't
Many of the young Lakota boys were sexually abused by a Catholic Prefect.
The terror, fright, and numbing of this experience has stuck with him to this day.
The abuse was so bad that by the eighth grade, Tony had started drinking heavily.
Because of this, Tony was expelled from the Holy Rosary Mission.
\vspace{5mm}

At the age of 15, Tony lied about his age, joined the Job Corps, and moved to Utah.
In Job Corps, Tony first encountered the concept of race.




\end{flushleft}
\end{document}
\}
