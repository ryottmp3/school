I lay limp on the back seat of my friend's truck,
speeding down the highway at 75 miles an hour.
My face is hot and wet with tears as I try to process what just happened.
My sister is dead.
Gone.
I will never see her again, and somehow I have to accept that.
I sit up and stare out the front windshield watching the lines on the road whisk by
and try to forget my pain, try to push it down so it can't hurt me.
As warm tears stream down my face, I realize that would be a fruitless
endeavor:
I needed to face this head on and let myself feel my feelings.
I collapse back into a heap on the seat, unable to control my bawling.
In that moment, I knew my life would never be the same.
 
I had just finished up at the county administration building getting my new truck registered.
The truck was legally mine, but in all reality it is my wife's truck.
I hadn't driven there, though. 
I biked there, like I do most places.
The line was long now and squeezing past the lines of tired-eyed people
who definitely did not want to be there was a task in and of itself.
Nevertheless, I squished past the last dreary-eyed person and hustled out the door.
My mom and grandma were in the parking lot, they wanted to meet with me before they took my sister
to the hospital.
She had been sick for a couple days, and she wasn't getting better.
She had been tested for strep, mono, and a battery of other sicknesses, 
they all had come back negative.
I needed to use my mom's car to go get my wife's truck, so she
dropped off her car with me before taking my sister to the hospital.
When I got to them, my mom and grandma were both crying and looked really worried.
My sister was immune compromised and they were worried because she was just so sick.
I was confident, however, that my sister was just fine and that she would pull
through this like she pulled through everything else.
This little girl had gone through 20 surgeries in her twelve years of life, 
I was sure she could handle a little flu.
I think my confidence was really a coping mechanism for my worries, though.
I was always worried about her. 
This little girl was the center of all of our lives. 
I opened the door to my grandma's car where my sister was sitting, half awake.
I looked at her little face and told her how much I loved her and that she would be okay
and that we're all rooting for her.
I took her up in my arms and gave her a big kiss on the forehead.
At the urging of my mom and grandma, I let go of her because they really needed to go to the hospital.
Looking back, I am glad that I took the time to tell her that, and to give her a hug and a kiss.
That was the last time I saw my twelve year-old sister.

That night, I went home with my mom's car and my wife went home with her new truck.
We had been living in our bus, but with Kymberlyn in the hospital, my mom wanted us to watch the pets.
That was fine by me, my mom's house is much closer to Mines than where the bus was parked, and it has 
running water.
So we went, packed up our four cats and our stuff and moved into my mom's house. 
We thought we would be there for a week at most.
It was midterms and I was neck-deep in studying for exams.
But I was worried.
It seemed like every couple minutes something else would go horribly wrong.
Her liver and kidneys were failing and she was in and out of consciousness.
That same night, she was life-flighted out to Omaha, NE because she just wasn't getting better.
She was eventually put on life support and even went blind.
Nothing seemed to help. 
The hospital told us it was Influenza B (We later found out it was influenza A) and that
they didn't know why her reaction was so bad.
Later we found out that Monument ER had given her a downer because she was being mean to the 
nurses and her reaction to that medicine is what caused everything to get so much worse so fast.
She held on strong over the weekend, while the hospital in Omaha tried everything they could to 
save her.
I had a small tour planned for spring break with my band.
With Kymberlyn as my courage, I decided to play two shows, one in Aberdeen SD and the other
in Rapid City\@ SD\@.
All throughout the Aberdeen show, Kymberlyn seemed to be getting a little bit better. 
She started on ECMO, a treatment like dialysis, and it seemed to be working.
The doctors told us that she had less than a 50\% chance of surviving that treatment.
Sadly, they were right.

That night, I was getting ready to say down to sleep after the concert when I got a call from my grandma.
It was the call I was dreading, the call I wish I could have never gotten.
She said that Kymberlyn had developed a brain bleed and that they were going to have to 
take her off of life support.
I cried uncontrollably.
My tears were like raindrops in a downpour.
I couldn't fathom this happening. 
I floated out of the house we were staying in onto the lawn and collapsed.
I pleaded with God, asking why He would do this, why He would cut her life so short.
I asked him to take me, the sinner, the one who has done so many horrible, horrible things, 
instead of taking this sweet, innocent young girl.
The stars were out that night as I lost myself in the grass, beating my chest and gnashing my teeth.
It's interesting how those actions seem so odd when you read about them in the Bible, but
when shit actually hits the fan, it's some sort of primal instinct.
My friends were with me, helping me get everything ready and get going. 
We thought we had time to get down to Omaha so I could see my sister one last time before she passed.
We started rushing down there as fast as possible in the truck.
Not even ten miles out of town, I get a video call from my mom. 
It's too late. 
They had to take her off now.
I got to see her face one last time and tell her how much I loved her and 
how much I cared about her and how sorry I was for being a mean older sibling.
I wept as I told her how I wished I spent more time with her and how I regret being a little bitch 
to her and how she really was a big reason for me keeping on with a positive life.
Without her and my mother, I don't think I would still be sober.
We had pulled over at this point and I was in the ditch, bawling with my sister on the phone.
I didn't want to hang up.
I didn't want to say goodbye.
I don't think anyone ever does.
But I had to.
I conference called in my wife, who was still at home, taking care of the animals.
She wept and wept and wept and wept.
After what felt like ages, my mom said they needed to go.
I collapsed again, under the bright starry sky in the middle of the highway.
I wished I could have been there, but my mom reminds me that my sister
would have been so grateful that I stayed behind to watch her puppies.
In the meantime, my friend Ti had grabbed my gun and holster and put it in the trailer.
We were all crying.
Kymberlyn meant the world to all of us.

In that moment, hundreds of miles away from where I was, 
my sister took her last breath surrounded by her closest family (save me and my wife)
while our mom sang her ``My Little Sunshine''.
My mom still rocks her ashes to sleep and sings that song every night, while I weep rooms away.
Our lives will never be the same.
