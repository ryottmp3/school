\documentclass[12pt]{article}

\usepackage{parskip}
\usepackage[margin=1in]{geometry}


\title{Federal Financial Aid Satisfactory Academic Progress Appeal}
\author{Harley 'Ryott' Glayzer}
\date{December 2023}

\begin{document}

\maketitle

\section*{Circumstances: Then and Now}
In my past semesters, I have struggled academically due to several limiting
factors that affected my performance. 
These factors ranged from being out of my control to being completely in my control.
However, I feel that I am now more apt to maintain satisfactory academic progress.
I have mitigated these limiting factors to the point that I feel comfortable managing them.
I will outline three factors outside my control and one factor within my control 
that have hindered my academic performance.

\paragraph{My Struggle with Homelessness}
To begin, I have experienced homelessness since I started school at SD Mines.
Until I met my now wife, I was living on friend's couches and in my truck. 
This, in addition to the other circumstances to be outlined later,
made it rather difficult to complete assignments and maintain satisfactory academic progress.
It also made it difficult to attend lectures and classes.
Now, I still live in a vehicle. 
However, that vehicle is a bus that is outfitted with access to electricity, water, and internet.
I have been able to complete work tasks from my Lab Assistant position under Dr. Schnee 
on the bus without many barriers.
In this regard, I plan to be very open and upfront with my professors, advisor, 
and Dr. Joe with any changes in my living situation.
If I end up in a situation without access to electricity or internet, I plan to use my
cubicle in the Dakota Building as a workspace to get all of my assignments done 
before I head home for the night.

\paragraph{My Struggle with Physical Disability}
Another limiting factor to my academic progress has been my struggle with chronic
illness and disability.
I have been diagnosed with Ehler's-Danlos Syndrome, Hypermobile Type:
a connective tissue disorder that causes chronic, debilitating pain in my joints;
frequent joint dislocations and subluxations; chronic fatigue; intense digestive issues;
problems with postural orthostatic tachycardia (dizziness and passing out when standing up);
incontinence; and many other symptoms that may occur, but I haven't dealt 
with yet (such as organ prolapse, intense bruising, miscellaneous heart problems 
and other internal organ issues).
In the past, I have had intense flare-ups that have kept me in bed (wherever that would be)
for up to days at a time. 
Recently, I have been making it a point to take care of my health and manage flare-up triggers. 
This includes exercising more, learning when to take a break from physical activity, 
some self-care routines, and doing what I can to eat healthier.
This has been aided by my acceptance into the SD SNAP Benefits program, the work 
of physicians at Oyate Health and beyond, and my choice to commute by bike (which 
seems to be the only exercise I can do without worsening my physical condition).
I can neither expect nor profess that I will not have another flare-up. 
I still have minor flare-ups that require me to take more Ibuprofen than is healthy.
However, I can state my plan concerning managing my disability and its effects on my academic progress.
I am working closely with the ADA coordinator to obtain the best accommodations
for my Ehlers-Danlos Syndrome, which may include lectures/slides/notes available
on days that I may not be able to attend class, and possibly attending lectures 
over ZOOM.
I am also doing everything I can to take care of myself and my body, and listening to its special needs.
I currently take the city bus part way and bike the rest of the way to Mines for 
my position as a Lab Assistant under Dr. Schnee and reside in my cubicle in the 
Dakota Building every day. 
Having this "home base" where I can do work and biking around campus (rather than walking)
has greatly improved my ability to manage flare-ups and get work done at the same time.
I cannot predict or change when flare-ups happen, but I can better communicate 
with my professors, advisor, and ADA coordinator when they do.


\paragraph{My Struggle with Severe ADHD}
A factor that has also led to my past unsatisfactory academic progress is my severe 
ADHD going untreated.
ADHD is often viewed as an excuse, or as trivial.
However, to those who struggle with the effects of severe ADHD, its effects on 
executive function are tangible in poor academic performance and disorganization
in all areas of life, which often leads to other debilitating circumstances.
Until recently, I had lost medical insurance and had no access to ADHD medication.
This inhibits my ability to organize my thoughts and thus my life. 
This had a part in my struggles to turn in assignments and focus on the
tasks at hand.
Now, I am medicated through Oyate Health and have state medical insurance. 
Since starting medication, I have seen significant improvement in my ability 
to get work done efficiently and thoroughly while keeping focus.
This has applied to all of the steps that I have done to get back into SD Mines,
as well as the work that I do as a Lab Assistant under Dr. Schnee.
As an aside, I have been medicated throughout the tour that I went on in Fall 2023,
and was able to maintain the same level of getting work done as being in the office.
This would not have been possible at any earlier, unmedicated point in my life. 
The tour would have proved itself to be a great distraction that kept me from getting
any work done.
I plan to continue my medication indefinitely, as well as work closely with the 
ADA coordinator on accommodations to benefit me.
Even though I have proven to thrive in an environment where programming is a 
core part of tasks, I have often struggled with the online assignment structures 
that are now commonplace in our university. 
This stems from the struggles of focus and organization that are only exacerbated
by the dopamine-centric internet 
(ADHD is on some level a dopamine deficiency).
To help with this, I am trying to get as many assignments as I can on paper.
I feel that this will help eliminate any issues I have after medication for my ADHD.
All in all, I feel that medication and upfront communication with the ADA coordinator, 
my professors, and my advisor are key steps in my plan to manage my ADHD.

%\paragraph{My Struggle with Addiction}
%I have, in past semesters, struggled with addiction and substance use. 
%I put my addiction before my academics, and I paid for that choice.
%However, I am in recovery and have been sober for about 18 months. 
%I have since been in situations where the substance had been available to me 
%and have used coping mechanisms I have gained from online sobriety forums and
%meeting with others with similar issues to avoid relapse.
%I do not intend to ever touch the substance again, and I have a large support
%system to help me with that.
%I will seek help if I relapse and be open with my advisor and any 
%other necessary parties in that search for help. 
%However, my addiction to substances hurt a lot of aspects of my life 
%(including this one), and I intend to stand strong in opposition to substance.

\paragraph{My Struggle with Maturity}
To be frank, I have struggled with maturity.
I came to SD Mines after skating through high school on test scores alone.
I was immature and dumb in many ways.
As time goes by, I am learning and maturing in my way.
Having to attempt to pay for my classes has taught me that I was taking my 
aid for granted.
Financial aid is a privilege and an enormous help for students like me, who frankly
likely couldn't make it without aid.
I neglected my aid by not working closely with the ADA, %falling into addiction,
not communicating my housing struggles with resources at the school, who I'm sure would have 
helped me in some regard there, and by not being proactive in my academics.
That neglect was the result of blatant immaturity and I can recognize that now. 
I can also recognize where situations are out of my control. 
However, I can also recognize where I can be more proactive in those situations 
and help others help me.
I feel that I have matured significantly since I lost my financial aid and that 
I am mature enough to handle the responsibility that comes with it.
I am steadfast in my intentions to continue maturing and learning.


\section*{Plans: Going Forward in Academia}
In the previous section, I outlined a few plans for how I am going to manage my
limiting factors regarding academic performance.
In this section, I will expand upon those plans.

\paragraph{Transportation and Attendance}
I plan to continue utilizing my bicycle and public transportation to attend classes.
On days when the trip is too much on a bike, 
I will either use my truck or solely public transit.
I plan to use mobility aids as necessary.
I own a walker and a cane and am working on obtaining an ambulatory wheelchair.
If I cannot attend a class or lecture, I will notify 
my professors, my academic advisor, the ADA coordinator, and Dr. Joe by email as
soon as I have the opportunity.
In that case, I will follow through with whatever suggestions or plans are set 
in place by my ADA plan and the discretion of the professor.

\paragraph{Student Work}
I feel that my work as a Lab Assistant under Dr. Schnee has positively impacted 
my performance and work ethic.
I plan to continue working under Dr. Schnee as a Lab Assistant while taking courses.

\paragraph{ADA Accomodations}
In the past, I have neglected to take full advantage of the accommodations afforded to me 
by the Americans with Disabilities Act.
This has led to poor communication and not receiving the help I need.
I plan on working closely with the ADA coordinator and my professors to ensure
positive academic performance even through my struggles with disability.

\paragraph{Tutoring}
In past semesters, I have neglected to utilize the tutoring available at SD Mines, 
due in part to attendance issues, and part to ego.
I have realized that not seeking help because of ego is pointless and dumb.
I plan on utilizing the tutoring programs available to the full extent I need.

\paragraph{Health}
I plan on continuing all medication I am currently prescribed, as well as 
regularly attending and scheduling doctor's appointments.
I plan on following all suggestions set forth by my physicians and specialists.
I will let appropriate people know when I must miss class due to a medical appointment.
I plan on continuing to do what I can to eat and live healthy, and be honest and 
ask for help when I can't.
I will be transparent and upfront about any issues that arise regarding my academic performance 
because of my health.

\paragraph{Communication}
I have often struggled with communication in all facets of life. 
However, I feel that I am learning and growing in my communication skills.
I intend to continue working on those communication skills by being upfront and 
proactive in my emails with my professors, advisor, ADA coordinator, and all other 
necessary parties.
I feel that I have shown tangible improvements in this regard which can be observed
through my emails.

\paragraph{Credits and Classes}
Through experience, I have seen that I have a hard time handling the full 
15-hour course load that is common for students in my major.
I intend to take somewhere between six and twelve credits each semester, depending
on classes available and the workload of courses.
I have discussed this with my academic advisor, Michael Dowding.
Currently, I am registered for MATH 125 (Calc II) and CHEM 112/L 
(General Chemistry) for Spring 2024, and plan on taking PHYS 211/L and MATH 225
in the Fall.

%\paragraph{Sobriety}
%I have removed all people from my life that were related to my addiction and could
%prove a negative influence in that regard.
%I utilize and plan to continue utilizing my large support system to keep myself sober.
%I left a tour in the middle of the tour because a bandmate started using the substance 
%and I was not going to put myself in that situation.
%My academics are much more important to me than getting high.
%I plan to communicate all risks relating to academics with my advisor, and if I were 
%to have struggles with a lapse or relapse, reach out to on-campus and off-campus resources.
%I plan to utilize the counseling service at SD Mines and be upfront about everything.



\section*{Conclusion}
Overall, I feel that I am ready to be eligible for federal financial aid.
If medical documentation is desired, it can be provided.
However, there may be a wait time on the part of the medical offices
in regard to medical release forms.













\end{document}

