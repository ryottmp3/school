\documentclass[12pt, letterpaper]{article}
\usepackage[includehead, margin=0.75in]{geometry}
\usepackage{fancyhdr}
\usepackage[british]{babel}
\usepackage{amsmath}
\usepackage{amsthm}
\usepackage{amssymb}
\usepackage{amstext}
\usepackage{soul}
\usepackage{color}
\usepackage{colortbl}
\usepackage{siunitx}
\usepackage{textgreek}
\usepackage{xspace}
\usepackage{lastpage}
\usepackage{lipsum}
\definecolor{error}{rgb}{255,255,0}
\newcommand{\degree}{\ensuremath{{}^{\circ}}\xspace}

\begin{document}

\fancypagestyle{plain}{%
\fancyhf{}%
\fancyhead[L]{MATH 125---M03 \\ Dr. May}
\fancyhead[R]{H. Ryott Glayzer \\ \today}%
\fancyhead[C]{Notes \\ \textit{Convergence of Taylor Series}}%
\fancyfoot[L]{\textsc{Notes \today}}%
\fancyfoot[C]{\thepage~of~\pageref{LastPage}}
\fancyfoot[R]{\textsc{Calc II with Dr. May}}
}


\title{10.9: Convergence of Taylor Series}
\author{H. Ryott Glayzer}
\date{\today}


\maketitle


%%%%%%%%%%%%%%%%%%%%%%%%%%%%%%%%%%%%%%%%%%%%%%%%%%%%%%%%%%%%%%%%%%%%%%%%%%%%%%%
A Taylor Series for $f(x)$ centered at the point $a$ can be defined as
\[
	\sum_{n=0}^{\infty} \frac{f^{(n)}{(x)}}{n!}{(x-a)}^{n}
\]

%%%%%%%%%%%%%%%%%%%%%%%%%%%%%%%%%%%%%%%%%%%%%%%%%%%%%%%%%%%%%%%%%%%%%%%%%%%%%%%
\section{Example 1}
Find the Taylor series for $f(x) = e^x$ centered at $a=0$.

\[
	\sum_{n=0}{\infty} \frac{x^n}{n!} 
\]
For what $x$ does $\sum\frac{x^n}{n!}$ converge?
\[
	\lim_{n\to\infty}\left| \frac{a_{n+1}}{a_n}  \right|
	=
	\lim_{n\to\infty} \left|\frac{x^{n+1}}{(n+1)!} \times \frac
\]

%%%%%%%%%%%%%%%%%%%%%%%%%%%%%%%%%%%%%%%%%%%%%%%%%%%%%%%%%%%%%%%%%%%%%%%%%%%%%%%
\section{Example 2}
Taylor series for $f(x) = \log x$ centered at $a=1$

\[
	f(x) = \log x \;
	f(1) = 0 \\
\]
\[
	f'(x) = \frac{1}{x} \;
	f'(1) = 1 \\
\]
\[
	f''(x) = -x^-2 \;
	f''(1) = -1 \\
\]
\[
	f'''(x) = 2x^-3 \;
	f'''(1) = 2 \\
\]
\[
	f''''(x) = -3\times2x^-4 \;
	f''''(1) = -6\\
\]


\[
	f^{(n)}(1) = (-1)^{n+1}(n-1)!
\]
Thus
\[
	\sum_{n=1}{\infty} \frac{(-1)^{n+1}(n-1)!}{n!}(x-1)^n
\]

Does it converge?

Because of the exponentials, it is good to use the ratio test
\[
	\lim_{n\to\infty} \left| \frac{(x-1)^{n+1}}{n+1}\times\frac{n}{(x-1)^n} \right|
\]

\[
	\lim_{n\to\infty} \left| \frac{(x-1)n}{n+1}
	\right| = \left| x-1 \right| < 1
\]
\[ 
	-1 < x-1 < 1
\]
For $0<x<2$, this series converges absolutely.


\subsection{Test Endpoints of absolute convergence}%
Test the endpoints $x=0$
\[
	\sum_{n=1}^{\infty} \frac{(-1)^{n+1}(-1)^n}{n}
\]
\[
	= -\sum_{n=1}^{\infty} \frac{(-1)^n(-1)^n}{n}
\]
\[
	= -\sum_{n=0}^{\infty}\frac{1}{n}
\]
Thus diverges at $x=0$\\

Test for $x=2$: Converges by alternating series test.\\

Taylor Series converges for $0<x<=2$



%%%%%%%%%%%%%%%%%%%%%%%%%%%%%%%%%%%%%%%%%%%%%%%%%%%%%%%%%%%%%%%%%%%%%%%%%%%%%%%
\section{Quiz Review}

I received a score of $\frac{15}{15}$ on my Chapter 10 Part 1 quiz.


%%%%%%%%%%%%%%%%%%%%%%%%%%%%%%%%%%%%%%%%%%%%%%%%%%%%%%%%%%%%%%%%%%%%%%%%%%%%%%%
\section{Homework Assignment}

The homework for today's lesson is 5 problems from chapter 10, section 9:
Convergence of Taylor Series, on page 647 in the book.

















\end{document}





%%%%%%%%%%%%%%%%%%%%%%%%%%%%%%%%%%%%%%%%%%%%%%%%%%%%%%%%%%%%%%%%%%%%%%%%%%%%%%%
%%%                                 Copypasta                               %%%
%%%%%%%%%%%%%%%%%%%%%%%%%%%%%%%%%%%%%%%%%%%%%%%%%%%%%%%%%%%%%%%%%%%%%%%%%%%%%%%

% Table for questions with multiple parts

% \begin{center}
% 	\begin{tabular}{|c|c|c|c|c|}
% 		\hline
% 		_ & _ & _ & _ & _ \\
% 		\hline
% 		_ & _ & _ & _ & _ \\
% 		\hline
% 	\end{tabular}
% \end{center}
