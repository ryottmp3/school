\documentclass[12pt, letterpaper]{article}
\usepackage[includehead, margin=0.75in]{geometry}
\usepackage{fancyhdr}
\usepackage[british]{babel}
\usepackage{amsmath}
\usepackage{amsthm}
\usepackage{amssymb}
\usepackage{amstext}
\usepackage{soul}
\usepackage{color}
\usepackage{colortbl}
\usepackage{siunitx}
\usepackage{textgreek}
\usepackage{xspace}
\usepackage{lastpage}
\usepackage{lipsum}
\definecolor{error}{rgb}{255,255,0}
\newcommand{\degree}{\ensuremath{{}^{\circ}}\xspace}

\begin{document}

\fancypagestyle{plain}{%
\fancyhf{}%
\fancyhead[L]{MATH 125---M03 \\ Dr.\ May}
\fancyhead[R]{H. Ryott Glayzer \\ \today}%
\fancyhead[C]{Notes \\ \textit{Infinite Series}}%
\fancyfoot[L]{\textsc{Infinite Series Notes}}%
\fancyfoot[C]{\thepage~of~\pageref{LastPage}}
\fancyfoot[R]{\textsc{Calc II with Dr.\ May}}
}


\title{Infinite Sequences and Series}
\author{H. Ryott Glayzer}
\date{\today}


\maketitle

\[
	\pi = \sum_{{n=0}}^{\infty}{\frac{(-1)^{n}}{2n+1}}
\]

\section{Infinite Sequences}

A sequence is a list of numbers in a given order, such as:
\[
	a_1, a_2, a_3, a_4, a_5, \ldots , a_n
\]

These are often represented sort of like functions:
\[
	a_n = \sqrt{n},\qquad b_n = (-1)^{n+1}\frac{1}{n}
\]





\section{Partial Sums of Sequences}































\end{document}





%%%%%%%%%%%%%%%%%%%%%%%%%%%%%%%%%%%%%%%%%%%%%%%%%%%%%%%%%%%%%%%%%%%%%%%%%%%%%%%
%%%                                 Copypasta                               %%%
%%%%%%%%%%%%%%%%%%%%%%%%%%%%%%%%%%%%%%%%%%%%%%%%%%%%%%%%%%%%%%%%%%%%%%%%%%%%%%%

% Table for questions with multiple parts

% \begin{center}
% 	\begin{tabular}{|c|c|c|c|c|}
% 		\hline
% 		_ & _ & _ & _ & _ \\
% 		\hline
% 		_ & _ & _ & _ & _ \\
% 		\hline
% 	\end{tabular}
% \end{center}
