%%% Homework Latex Template for Physics 211
%%% Copyright © 2024  H. Ryott Glayzer
%%% MIT License

\documentclass[12pt, letterpaper]{article}

\usepackage[T1]{fontenc}
\usepackage[utf8]{inputenc}

\title{Vibrations, Waves, \& Optics Notes}
\author{H. Ryott Glayzer}
\date{\today}


\begin{document}
\maketitle
\section{Review}
\[
\lambda = \frac{v}{f}
\]

Frequency, rate at which oscillation repeats at a given spot,
must be same from one side of boundary to other by continuity.

If velocity increases (decreases) at boundary, wavelength
(distance between peaks) must increase (decrease) proportionally.

For each point on wavefront launches a circular wavelet.
Find new wavefront from tangent to extremes of wavelets,
because everything else ends up cancelling.

This is Huygen's Principal.

\section{New Shit}

When thinking of angle changes on wavefronts, it is useful
to think of rolling a can on concrete onto grass.
Because one edge of the can will touch grass first, the
can will turn at such an angle.

These examples and principles are reversible.


For a wavefront launched in huygens sense at same time at right angles














\end{document}








