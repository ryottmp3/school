\documentclass[11pt, letterpaper]{article}
\usepackage[includehead, margin=0.75in]{geometry}
\usepackage{fancyhdr}
\usepackage[british]{babel}
\usepackage{sansmathfonts}
\usepackage{soul}
\usepackage{color}
\usepackage{colortbl}
\usepackage{lewis}
\usepackage{siunitx}
\usepackage{textgreek}
\usepackage{mhchem}
\usepackage{modiagram}
\usepackage{tikzorbital}
\usepackage{chemfig}
\usepackage{xspace}
\definecolor{error}{rgb}{255,255,0}
\newcommand{\degree}{\ensuremath{{}^{\circ}}\xspace}
\renewcommand{\familydefault}{\sfdefault}

\begin{document}

\fancypagestyle{plain}{
\fancyhf{}
\fancyhead[L]{CHEM 112 - M01 \\ Dr. Moulder}
\fancyhead[R]{H. Ryott Glayzer \\ \today}
\fancyhead[C]{Notes \\ \textit{Acids and Bases}}
}


\title{Notes Acids and Bases}
\author{H. Ryott Glayzer}
\date{\today}


\maketitle


\section*{Notice of ADA Accommodation and Methods}
I have an ADA accommodation to do my assignment on paper.
This document is a utilization of that accommodation.
This assignment will utilize questions from the textbook,
\textit{Chemistry: Atoms First, 2e}, to practice the skills
and learning objectives for this class.

Learning Objectives 3 \& 4

\section*{Common Acids and Bases}
\subsection*{The pH Scale}
Every substance falls between 0 and 14 on the pH scale.
It measures the speed at which an acid separates into its component
electrolytes.
\subsection*{Different Acid/Base Definitions}
Arrhenius Acid: dissociates into [\ce{H+}] in a solution of water.

Bronsted-Lowry Acid: Proton \ce{H+} donor

Lewis Acid: Accepts Electron pair to form a bond

Arrhenius Base: dissociates into \ce{OH-} in a solution of water

Bronsted-Lowry Base: Proton \ce{H+} acceptor

Lewis Base: Gives electron pair to form a bond

\subsection*{Conjugate Acids and Bases}
An whenever an acid or base exists, there is always a conjugate to the chemical.
It is always a pair, it doesn't need to be something generally defined as such.


\subsection*{Strong acids to memorize}

\begin{itemize}
	\item \ce{Hbr}
	\item \ce{HCl}
	\item
	\item
	\item
\end{itemize}

\subsection*{Weak Acids}

\begin{itemize}
	\item \ce{HF}
	\item \ce{CH3CO2H}
	\item 
	\item
\end{itemize}

Use a weak acid to neutralize a weak base, and a strong acid to neutralize a strong base.

\subsection*{Bronsted-Lowry Bases}

A substance that will dissolve in water to yield hydroxide ions, \ce{OH-}.
Most common bases ore ionic compounds composed of alkalai or alkaline earch metal cations
combined with the hydroxide ion.
Examples include \ce{NaOH}, \ce{KOH}, \ce{Ca(OH)2}, \ce{Ba(OH)2}
These bases, along with other hydroxides completely dissociate in water are considered 
\textbf{Strong Bases}.

\subsection*{Weak Bases}
Compounds that react only partially in water are called \textbf{Weak Bases}.
Ammonia is one such base.

\subsection*{Polyprotic Acids}
An acid can have more than one hydrogen to be donated.
One example is H3PO4.
These are generally only strong acids for their first proton definition.



\section*{Stoichiometry}
Stoichiometry is the art of balancing chemical equations and following recipies.
This will be \textbf{very} important in this class.

\ce{2N2 + 5O2 -> 2N2O5 }

\subsection*{Equations for Ionic Reactions}
Many chemical reachions take glace in aqueous solution.
Molecular equations don't explicitly represent the ionic species that are present in solution.
When ions are involved, we need to keep track of charges

A Net Ionic equation is the full ionic equation without the spectator ions that don't change 
in the reaction. 











\section*{Redox Reactions}






















\end{document}





%%%%%%%%%%%%%%%%%%%%%%%%%%%%%%%%%%%%%%%%%%%%%%%%%%%%%%%%%%%%%%%%%%%%%%%%%%%%%%%
%%%                                 Copypasta                               %%%
%%%%%%%%%%%%%%%%%%%%%%%%%%%%%%%%%%%%%%%%%%%%%%%%%%%%%%%%%%%%%%%%%%%%%%%%%%%%%%%

% Table for questions with multiple parts

% \begin{center}
% 	\begin{tabular}{|c|c|c|c|c|}
% 		\hline
% 		_ & _ & _ & _ & _ \\
% 		\hline
% 		_ & _ & _ & _ & _ \\
% 		\hline
% 	\end{tabular}
% \end{center}
