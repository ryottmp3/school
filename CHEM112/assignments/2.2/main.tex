\documentclass[11pt, letterpaper]{article}
\usepackage[includehead, margin=0.75in]{geometry}
\usepackage{fancyhdr}
\usepackage[british]{babel}
\usepackage{sansmathfonts}
\usepackage{amssymb}
\usepackage{soul}
\usepackage{color}
\usepackage{colortbl}
\usepackage{lewis}
\usepackage{siunitx}
\usepackage{textgreek}
\usepackage{mhchem}
\usepackage{modiagram}
\usepackage{tikzorbital}
\usepackage{chemfig}
\usepackage{xspace}
\definecolor{error}{rgb}{255,255,0}
\newcommand{\degree}{\ensuremath{{}^{\circ}}\xspace}
\renewcommand{\familydefault}{\sfdefault}

\begin{document}

\fancypagestyle{plain}{%
\fancyhf{}
\fancyhead[L]{CHEM 112 - M01 \\ Dr. Moulder}
\fancyhead[R]{H. Ryott Glayzer \\ 1st April 2024}
\fancyhead[C]{Assignment \\ \textit{Section 2.2-3}}
}


\title{Section 2.2-3 Homework}
\author{H. Ryott Glayzer}
\date{1st April 2024}


\maketitle


\section*{Notice of ADA Accommodation and Methods}
I have an ADA accommodation to do my assignment on paper.
This document is a utilization of that accommodation.
This assignment will utilize questions from the textbook,
\textit{Chemistry: Atoms First, 2e}, to practice the skills
and learning objectives for this class.

\section*{Learning Objective 2.2.1 \& 2.2.2: Chapter 2.4}
\subsection*{2.4.37: Write a sentence that describes how to determine the number of
moles of a compound in a known mass of the compound if we know the molecular formula}

If we know the mass and the molecular formula of a sample, we can determine its moles
by deriving the molar mass via atomic mass units and dividing the mass of the sample 
by its molar mass.

\subsection*{2.4.39: Which contains the greatest mass of oxygen: 0.75 mol of ethanol,
0.60 mol of formic acid, or 1.0 mol of water?}

0.60 mol of formic acid contains 1.20 mol of oxygen, which is more than any other
chemical provided.

\subsection*{2.4.41: How are the molecular mass and molar mass of a compound related?}

The molecular mass of a compound is the mass in amu of the molecule, while the 
molar mass is the mass of one mole of the compound in grams. 
Generally, the numbers of these are the same, with different units.


\subsection*{2.4.43: Calculate the molar mass of each of the following:}
\subsubsection*{2.4.43.a: \ce{S8}}

256.48 g

\subsubsection*{2.4.43.b: \ce{C5H12}}

72.146 g

\subsubsection*{2.4.43.c: \ce{Sc2(SO4)3}}

378.10 g

\subsubsection*{2.4.43.d: \ce{CH3COCH3}}

58.078 g

\subsubsection*{2.4.43.e: \ce{C6H12O6}}

180.156 g



\subsection*{2.4.47: Determine the mass for each of the following:}
\subsubsection*{2.4.47.a: 0.0416 mol \ce{KOH}}

2.33 g

\subsubsection*{2.4.47.b: 10.2 mol \ce{C2H6}}

306 g

\subsubsection*{2.4.47.c: $1.6 \times 10^{-3}$ mol \ce{Na2SO4}}

0.20 g

\subsubsection*{2.4.47.d: $6.854 \times 10^{3}$ mol \ce{C6H12O6}}

$1,235 \times 10^{6}$ g

\subsubsection*{2.4.47.e: 2.86 mol \ce{Co(NH3)6Cl3}}

765 g



\section*{Learning Objective 2.2.3: Chapter 6.1}
\subsection*{6.1.1: What is the total mass in amu of carbon in each of the following?}
\subsubsection*{6.1.1.a: \ce{CH4}}
12.01 amu
\subsubsection*{6.1.1.b: \ce{CHCl3}}
12.01 amu
\subsubsection*{6.1.1.c: \ce{C12H10O6}}
144.12 amu
\subsubsection*{6.1.1.d: \ce{CH3CH2CH2CH2CH3}}
60.05 amu

\subsection*{6.1.3: Calculate the molecular or formula mass of each of the following:}
\subsubsection*{6.1.3.a: \ce{P4}}
123.88
\subsubsection*{6.1.3.b: \ce{H2O}}
18.016
\subsubsection*{6.1.3.c: \ce{Ca(NO3)2}}
164.10
\subsubsection*{6.1.3.d: \ce{CH3CO2H}}
60.052
\subsubsection*{6.1.3.e: \ce{C12H22O11}}
338.292

\section*{Learning Objective 2.2.4 \& 2.2.5: Chapter 6.2}
\subsection*{6.2.7: What information is needed to determine the molecular formula
from the empirical formula?}
The molecular formula can be obtained from the empirical formula if the molar mass is known,
or the number of moles of sample and mass of sample.

\subsection*{6.2.9: Determine the following to four sigfigs:}
\subsubsection*{6.2.9.a: Percent Composition of \ce{HN3}}
H: 2.342\% ; N: 97.66\%
\subsubsection*{6.2.9.b: Percent Composition of \ce{C6H2(CH3)(NO2)3}}
C: 37.01\% ; H: 2.219\% ; N: 18.50\% ; O: 42.26\%
\subsubsection*{6.2.9.c: Percent of \ce{SO4^{2-}} in \ce{Al2(SO4)3}}
84.23\%

\subsection*{6.2.11: Determine the percent water in \ce{CuSO4*5H2O} to three sigfigs}
36.1\%
\subsection*{6.2.15: Dichloroethane, a compound that is often using for dry cleaning,
contains carbon (24.3\%), hydrogen (4.1\%) and chlorine with a molar mass of 99 g/mol}
\ce{C2H4Cl2}

\section*{Learning Objective 2.2.6: Chapter 6.3}
\subsection*{6.3.19: Explain what changes and what stays the same when 1.00L of a 
solution of \ce{NaCl} is diluted to 1.80L}
The number of moles of salt stays the same, while the molarity and concentration changes.

\subsection*{6.3.21: A 200-mL sample and a 400-mL sample of a solution of salt 
have the same molarity. How are they related?}

The samples both have the same concentration, but the 400 mL sample has twice the
amount of salt as the 200 mL solution.


\subsection*{6.3.25: Consider this question: What is the mass of solute in 200.0L
of a 1.556-M solution of \ce{KBr}?}

$3.703 \times 10^{4}$ grams

\subsection*{6.3.29: Consider this question: What is the molarity of \ce{HCl} 
if 35.23 mL of a solution of \ce{HCl} contain 0.3366g of \ce{HCl}?}

0.2621 M

\subsection*{6.3.33: What volume of a 1.00 M \ce{Fe(NO3)3} solution can be 
diluted to prepare 1.00 L of a solution with a concentration of 0.250 M?}

250 mL

\section*{Learning Objective 2.2.7 - 2.2.9: Chapter 7.3}

\subsection*{7.3.47: Gallium Chloride is formed by the reaction of 2.6 L 
of a 1.44 M solution of HCl according to the following equation: 
\ce{2Ga + 6 HCl -> 2GoCl3 + 3H2}}

1.25 mol \ce{GaCl3}, $2.2 \times 10^{2}$ g \ce{GaCl3}

\subsection*{7.3.51: Carborundum is silicon carbide, \ce{SiC}, a very hard material
used as an abrasive on sandpaper and in other applications.
It is prepared by the reaction of pure sand, \ce{SiO2},
with carbon at high temperature. Carbon monoxide, \ce{CO},
is the other product of this reaction.
Write the balanced equation for the reaction,
and calculate how much \ce{SiO2} is required to produce 3.00 kg of \ce{SiC}.}

\ce{SiO2 + 3C -> SiC + 2CO}, 4.50 kg \ce{SiO2}

\subsection*{7.3.55: A compact car gets 37.5 miles per gallon on the highway.
If gasoline contains 84.2\% carbon by mass and has a density of 0.8205 g/mL,
determine the mass of carbon dioxide produced during a
500-mile trip (3.785 liters per gallon).}

$1.28 \times 10^{5}$ g \ce{CO2}

\subsection*{7.3.59: The toxic pigment called white lead, \ce{Pb3(OH)2(CO3)2}, 
has been replaced in white paints by rutile, \ce{TiO2}. 
How much rutile (g) can be prepared from 379 g of an ore that contains 
88.3\% ilmenite (\ce{FeTiO3}) by mass?}

176 g \ce{TiO2}

\section*{Learning Objective 2.2.10 - 2.2.12: Chapter 7.4}

\subsection*{7.4.63: A student isolated 25 g of a compound following a 
procedure that would theoretically yield 81 g. What was his percent yield?}

31\% yield

\subsection*{7.4.67: Toluene, \ce{C6H5CH3}, is oxidized by air under carefully 
controlled conditions to benzoic acid, \ce{C6H5CO2H}, 
which is used to prepare the food preservative sodium benzoate, \ce{C6H5CO2Na}. 
What is the percent yield of a reaction that converts 1.000 kg of toluene 
to 1.21 kg of benzoic acid?}

91.3\% yield

\subsection*{7.4.71: Outline the steps needed to determine the
limiting reactant when 0.50 mol of \ce{Cr} and 0.75 mol of \ce{H3PO4}
react according to the following chemical equation.
\ce{2Cr + 2H3PO4 -> 2CrPO4 + 3H2}.  Determine the limiting reactant.}

moles of \ce{Cr} to moles of \ce{H3PO4}, compare amt of Cr to amt of acid.
\ce{Cr} is limreag.

\subsection*{7.4.75: How many molecules of the sweetener saccharin can 
be prepared from 30 C atoms, 25 H atoms, 12 O atoms, 8 S atoms, and 14 N atoms?}

four molecules

\section*{Learning Objective 2.2.13 - 2.2.14: Chapter 7.5}


\subsection*{7.5.79: Titration of a 20.0-mL sample of acid rain 
required 1.7 mL of 0.0811 M \ce{NaOH} to reach the end point. 
If we assume that the acidity of the rain is due to the 
presence of sulfuric acid, what was the concentration 
of sulfuric acid in this sample of rain?}

$3.4 \times 10^{-3}$ M \ce{H2SO4}

\subsection*{7.5.83: A sample of gallium bromide, \ce{GaBr3}, 
weighing 0.165 g was dissolved in water and treated with silver nitrate, 
\ce{AgNO3}, resulting in the precipitation of 0.299 g \ce{AgBr}. 
Use these data to compute the \%\ce{Ga} (by mass) \ce{GaBr3}.}

22.4\%

\subsection*{7.5.87: What volume of 0.600 M \ce{HCl} is required to react 
completely with 2.50 g of sodium hydrogen carbonate?
\ce{NaHCO3 (aq) + HCl (aq) -> NaCl (aq) + CO2 (g) + H2O (l)}}

49.6 mL

\subsection*{7.5.91: A sample of solid calcium hydroxide, \ce{Ca(OH)2}, 
is allowed to stand in water until a saturated solution is formed. 
A titration of 75.00 mL of this solution with $5.00 \times 10^{-2}$
M \ce{HCl} requires 36.6 mL of the acid to reach the end point.
\ce{Ca(OH)2 (aq) + 2HCl (aq) -> CaCl2 (aq) + 2H2O (l)} 
What is the molarity?}

0.0122 M

\subsection*{7.5.95: The reaction of \ce{WCl6} with \ce{Al} at \~400 \degree C 
gives black crystals of a compound containing only tungsten and chlorine. 
A sample of this compound, when reduced with hydrogen, 
gives 0.2232 g of tungsten metal and hydrogen chloride, 
which is absorbed in water. Titration of the hydrochloric acid thus 
produced requires 46.2 mL of 0.1051 M \ce{NaOH} to reach the end point. 
What is the empirical formula of the black tungsten chloride?}

\ce{WCl4}

\section*{Learning Objective 2.3.1 - 2.2.5: Chapter 9.1}


\subsection*{9.1.1: A burning match and a bonfire may have the same temperature,
yet you would not sit around a burning match on a fall evening to stay warm. Why not?}

Theres not as much heat (heat is extensive).

\subsection*{9.1.5: Calculate the heat capacity, in joules
and in calories per degree, of 45.8 g of 
nitrogen gas and 1.00 pound of aluminum metal}


\begin{center}
	\begin{tabular}{|c|c|c|c|}
		\hline
		\ce{N2} & \ce{N2} & Al & Al \\
		\hline
		47.6 J/\degree C & 11.38 cal/\degree C & 407 J/\degree C & 97.3 cal/\degree C \\
		\hline
	\end{tabular}
\end{center}

\subsection*{9.1.9: If 14.5 kJ of heat were added to 485 g of
liquid water, how much would its temperature increase?}

7.15 \degree C

\subsection*{9.1.13: Most people find waterbeds uncomfortable unless the 
water temperature is maintained at about 85 \degree F. 
Unless it is heated, a waterbed that contains 892 L of water 
cools from 85 \degree F to 72 \degree F in 24 hours. 
Estimate the amount of electrical energy required over 
24 hours, in kWh, to keep the bed from cooling. 
Note that 1 kilowatt-hour (kWh) = $3.6 \times 10^{6}$ J, 
and assume that the density of water is 1.0 g/mL 
(independent of temperature). 
What other assumptions did you make? 
How did they affect your calculated result 
(i.e., were they likely to yield “positive” or “negative” errors)?
}

Assumptions:
\begin{itemize}
	\item Water is 1.0 g \times cm$^{-3}$
	\item Water will be the only thing heated
	\item energy to heat water doesn't change with temperature, or is at
		least negligible at the temperatures necessary
\end{itemize}

7.47 kWh is required.


\section*{Learning Objective 2.3.6 - 2.3.7: Chapter 9.2}

\subsection*{9.2.17: Would the amount of heat absorbed by the dissolution 
in Example 9.6 appear greater, lesser, or remain the same if the heat 
capacity of the calorimeter were taken into account? Explain your answer.}

greater, as the heat capacity of the calorimeter will account for heat put into
the solution.

\subsection*{9.2.21: The temperature of the cooling water as it leaves the 
hot engine of an automobile is 240 \degree F. After it passes through the 
radiator it has a temperature of 175 \degree F. Calculate the amount of heat 
transferred from the engine to the surroundings by one gallon of water with 
a specific heat of 4.184 J/g \degree C.}

$5.7 \times 10^{2}$ kJ

\subsection*{9.2.25: Dissolving 3.0 g of \ce{CaCl2(s)} in 150.0 g of water 
in a calorimeter (Figure 9.12) at 22.4 \degree C causes the temperature to 
rise to 25.8 \degree C. What is the approximate amount of heat involved 
in the dissolution, assuming the specific heat of the resulting solution is 
4.18 J/g \degree C? Is the reaction exothermic or endothermic?}

-2.2 kJ, Exothermic since heat is negative

\subsection*{9.2.29: If the 3.21 g of \ce{NH4NO3} in Example 9.6 were 
dissolved in 100.0 g of water under the same conditions, how much would the 
temperature change? Explain your answer.}

22.6 K.\\ 
$m_{water} \approx m_{solution} \land C_{water} \approx C_{solution} \therefore mol_{i} = mol_{f} \times 2 \implies T_{i} = \frac{1}{2} \times T_{f}$

\subsection*{9.2.33: The amount of fat recommended for someone with a daily
diet of 2000 Calories is 65 g. What percent of the calories in this diet 
would be supplied by this amount of fat if the average number of Calories for fat is
9.1 Calories/g?}

30\%

\subsection*{9.2.37: A serving of a breakfast cereal contains 3 g of protein, 
18 g of carbohydrates, and 6 g of fat. What is the Calorie content of a 
serving of this cereal if the average number of Calories for fat is 
9.1 Calories/g, for carbohydrates is 4.1 Calories/g, 
and for protein is 4.1 Calories/g?}

$1.4 \times 10^{2}$ Calories

\section*{Learning Objective 2.3.8 - 2.3.12: Chapter 9.3}


\subsection*{9.3.41: Calculate the enthalpy of solution 
($\Delta$H for the dissolution) per mole of \ce{NH4NO3} under the 
conditions described in Example 9.6.}

25 kJ/mol

\subsection*{9.3.45: How much heat is produced by burning 4.00 
moles of acetylene under standard state conditions?}

5204.4 kJ

\subsection*{9.3.49: When 2.50 g of methane burns in oxygen, 125 kJ 
of heat is produced. What is the enthalpy of combustion per mole 
of methane under these conditions?}

-802 kJ/mol

\subsection*{9.3.53: Homes may be heated by pumping hot water through 
radiators. What mass of water will provide the same amount of heat 
when cooled from 95.0 to 35.0 \degree C, as the heat provided when 
100 g of steam is cooled from 110 \degree C to 100 \degree C?}

7.43 g

\subsection*{9.3.57: How many kilojoules of heat will be released 
when exactly 1 mole of manganese, Mn, is burned to form 
\ce{Mn3O4(s)} at standard state conditions?}

459.6 kJ

\subsection*{9.3.61: From the molar heats of formation in Appendix G, 
determine how much heat is required to evaporate one mole of water:
\ce{H2O (l) -> H2O (g)}}

44.01 kJ/mol

\subsection*{9.3.65: Calculate ΔH for the process \ce{Hg2Cl2 (s) -> 2Hg (l) + Cl2 (g)} 
from the following information:
\ce{Hg (l) + Cl2 (g) -> HgCl2 (s) \Delta H = −224 kJ}
\ce{Hg (l) + HgCl2 (s) -> Hg2Cl2 (s) \Delta H=−41.2kJ}}

265 kJ

\subsection*{9.3.73: Calculate the enthalpy of combustion of butane, 
	\ce{C4H10 (g)} for the formation of \ce{H2O (g)} and \ce{CO2 (g)}. 
	The enthalpy of formation of butane is −126 kJ/mol.
}

-2660 kJ/mol

\subsection*{9.3.77: In the early days of automobiles, illumination at night was 
provided by burning acetylene, \ce{C2H2}. Though no longer used as auto headlamps, 
acetylene is still used as a source of light by some cave explorers. 
The acetylene is (was) prepared in the lamp by the reaction of water 
with calcium carbide, \ce{CaC2}:
\ce{CaC2 (s) + 2H2O (l) -> Ca(OH)2 (s) + C2H2 (g)}.
Calculate the standard enthalpy of the reaction.
The $\Delta$ H_{f}\degree of \ce{CaC2} is −15.14 kcal/mol.}

-122.8 kJ

\subsection*{9.3.83: Ethylene, \ce{C2H4}, a byproduct from the fractional
distillation of petroleum, is fourth among the 50 chemical compounds produced 
commercially in the largest quantities. About 80\% of synthetic ethanol is 
manufactured from ethylene by its reaction with water in the presence of a 
suitable catalyst.  \ce{C2H4 (g) + H2O (g) -> C2H5OH (l)}
Using the data in the table in Appendix G, calculate $\Delta$ H\degree for the reaction.}

-88.2 kJ

\section*{Learning Objective 2.3.13 - 2.3.14: Chapter 9.4}


\subsection*{9.4.93: How does the bond energy of \ce{HCl(g)} differ 
from the standard enthalpy of formation of \ce{HCl(g)}?}

Enthalpy of formation is a measurement of the energy associated with forming a compound while
the bond energy is a measurement of the energy associated with breaking the bonds.

\subsection*{9.4.97: Using the standard enthalpy of formation data in Appendix G, 
determine which bond is stronger: the P–Cl bond in \ce{PCl3 (g)} or in \ce{PCl5 (g)}?}

\ce{PCl5}


\subsection*{9.4.101: The lattice energy of LiF is 1023 kJ/mol, and 
the Li–F distance is 200.8 pm. NaF crystallizes in the same structure as 
LiF but with a Na–F distance of 231 pm. Which of the following values most 
closely approximates the lattice energy of NaF: 510, 890, 1023, 1175, or 4090 kJ/mol? 
Explain your choice.}

1175 kJ/mol, inverse square relationship

\end{document}





%%%%%%%%%%%%%%%%%%%%%%%%%%%%%%%%%%%%%%%%%%%%%%%%%%%%%%%%%%%%%%%%%%%%%%%%%%%%%%%
%%%                                 Copypasta                               %%%
%%%%%%%%%%%%%%%%%%%%%%%%%%%%%%%%%%%%%%%%%%%%%%%%%%%%%%%%%%%%%%%%%%%%%%%%%%%%%%%

% Table for questions with multiple parts

% \begin{center}
% 	\begin{tabular}{|c|c|c|c|c|}
% 		\hline
% 		_ & _ & _ & _ & _ \\
% 		\hline
% 		_ & _ & _ & _ & _ \\
% 		\hline
% 	\end{tabular}
% \end{center}
