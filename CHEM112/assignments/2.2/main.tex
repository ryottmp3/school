\documentclass[11pt, letterpaper]{article}
\usepackage[includehead, margin=0.75in]{geometry}
\usepackage{fancyhdr}
\usepackage[british]{babel}
\usepackage{sansmathfonts}
\usepackage{soul}
\usepackage{color}
\usepackage{colortbl}
\usepackage{lewis}
\usepackage{siunitx}
\usepackage{textgreek}
\usepackage{mhchem}
\usepackage{modiagram}
\usepackage{tikzorbital}
\usepackage{chemfig}
\usepackage{xspace}
\definecolor{error}{rgb}{255,255,0}
\newcommand{\degree}{\ensuremath{{}^{\circ}}\xspace}
\renewcommand{\familydefault}{\sfdefault}

\begin{document}

\fancypagestyle{plain}{
\fancyhf{}
\fancyhead[L]{CHEM 112 - M01 \\ Dr. Moulder}
\fancyhead[R]{H. Ryott Glayzer \\ \today}
\fancyhead[C]{Assignment \\ \textit{Section 2.2}}
}


\title{Section 2.2 Homework}
\author{H. Ryott Glayzer}
\date{\today}


\maketitle


\section*{Notice of ADA Accommodation and Methods}
I have an ADA accommodation to do my assignment on paper.
This document is a utilization of that accommodation.
This assignment will utilize questions from the textbook,
\textit{Chemistry: Atoms First, 2e}, to practice the skills
and learning objectives for this class.

\section*{Learning Objective 2.2.1 \& 2.2.2: Chapter 2.4}
\subsection*{2.4.37: Write a sentence that describes how to determine the number of
moles of a compound in a known mass of the compound if we know the molecular formula}

If we know the mass and the molecular formula of a sample, we can determine its moles
by deriving the molar mass via atomic mass units and dividing the mass of the sample 
by its molar mass.

\subsection*{2.4.39: Which contains the greatest mass of oxygen: 0.75 mol of ethanol,
0.60 mol of formic acid, or 1.0 mol of water?}

0.60 mol of formic acid contains 1.20 mol of oxygen, which is more than any other
chemical provided.

\subsection*{2.4.41: How are the molecular mass and molar mass of a compound related?}

The molecular mass of a compound is the mass in amu of the molecule, while the 
molar mass is the mass of one mole of the compound in grams. 
Generally, the numbers of these are the same, with different units.


\subsection*{2.4.43: Calculate the molar mass of each of the following:}
\subsubsection*{2.4.43.a: \ce{S8}}

256.48 g

\subsubsection*{2.4.43.b: \ce{C5H12}}

72.146 g

\subsubsection*{2.4.43.c: \ce{Sc2(SO4)3}}

378.10 g

\subsubsection*{2.4.43.d: \ce{CH3COCH3}}

58.078 g

\subsubsection*{2.4.43.e: \ce{C6H12O6}}

180.156 g



\subsection*{2.4.47: Determine the mass for each of the following:}
\subsubsection*{2.4.47.a: 0.0416 mol \ce{KOH}}

2.33 g

\subsubsection*{2.4.47.b: 10.2 mol \ce{C2H6}}

306 g

\subsubsection*{2.4.47.c: $1.6 \times 10^{-3}$ mol \ce{Na2SO4}}

0.20 g

\subsubsection*{2.4.47.d: $6.854 \times 10^{3}$ mol \ce{C6H12O6}}

$1,235 \times 10^{6}$ g

\subsubsection*{2.4.47.e: 2.86 mol \ce{Co(NH3)6Cl3}}

765 g



\section*{Learning Objective 2.2.3: Chapter 6.1}
\subsection*{6.1.1: What is the total mass in amu of carbon in each of the following?}
\subsubsection*{6.1.1.a: \ce{CH4}}
12.01 amu
\subsubsection*{6.1.1.b: \ce{CHCl3}}
12.01 amu
\subsubsection*{6.1.1.c: \ce{C12H10O6}}
144.12 amu
\subsubsection*{6.1.1.d: \ce{CH3CH2CH2CH2CH3}}
60.05 amu

\subsection*{6.1.3: Calculate the molecular or formula mass of each of the following:}
\subsubsection*{6.1.3.a: \ce{P4}}
123.88
\subsubsection*{6.1.3.b: \ce{H2O}}
18.016
\subsubsection*{6.1.3.c: \ce{Ca(NO3)2}}
164.10
\subsubsection*{6.1.3.d: \ce{CH3CO2H}}
60.052
\subsubsection*{6.1.3.e: \ce{C12H22O11}}
338.292

\section*{Learning Objective 2.2.4 \& 2.2.5: Chapter 6.2}
\subsection*{6.2.7: What information is needed to determine the molecular formula
from the empirical formula?}
The molecular formula can be obtained from the empirical formula if the molar mass is known,
or the number of moles of sample and mass of sample.

\subsection*{6.2.9: Determine the following to four sigfigs:}
\subsubsection*{6.2.9.a: Percent Composition of \ce{HN3}}
H: 2.342\% ; N: 97.66\%
\subsubsection*{6.2.9.b: Percent Composition of \ce{C6H2(CH3)(NO2)3}}
C: 37.01\% ; H: 2.219\% ; N: 18.50\% ; O: 42.26\%
\subsubsection*{6.2.9.c: Percent of \ce{SO4^{2-}} in \ce{Al2(SO4)3}}
84.23\%

\subsection*{6.2.11: Determine the percent water in \ce{CuSO4*5H2O} to three sigfigs}
36.1\%
\subsection*{6.2.15: Dichloroethane, a compound that is often using for dry cleaning,
contains carbon (24.3\%), hydrogen (4.1\%) and chlorine with a molar mass of 99 g/mol}
\ce{C2H4Cl2}

\section*{Learning Objective 2.2.6: Chapter 6.3}
\subsection*{6.3.19: Explain what changes and what stays the same when 1.00L of a 
solution of \ce{NaCl} is diluted to 1.80L}
The number of moles of salt stays the same, while the molarity and concentration changes.

\subsection*{6.3.21: A 200-mL sample and a 400-mL sample of a solution of salt 
have the same molarity. How are they related?}

The samples both have the same concentration, but the 400 mL sample has twice the
amount of salt as the 200 mL solution.


\section*{Learning Objective 2.2.7 - 2.2.9: Chapter 7.3}




\section*{Learning Objective 2.2.10 - 2.2.12: Chapter 7.4}




\section*{Learning Objective 2.2.13 - 2.2.14: Chapter 7.5}
































\end{document}





%%%%%%%%%%%%%%%%%%%%%%%%%%%%%%%%%%%%%%%%%%%%%%%%%%%%%%%%%%%%%%%%%%%%%%%%%%%%%%%
%%%                                 Copypasta                               %%%
%%%%%%%%%%%%%%%%%%%%%%%%%%%%%%%%%%%%%%%%%%%%%%%%%%%%%%%%%%%%%%%%%%%%%%%%%%%%%%%

% Table for questions with multiple parts

% \begin{center}
% 	\begin{tabular}{|c|c|c|c|c|}
% 		\hline
% 		_ & _ & _ & _ & _ \\
% 		\hline
% 		_ & _ & _ & _ & _ \\
% 		\hline
% 	\end{tabular}
% \end{center}
