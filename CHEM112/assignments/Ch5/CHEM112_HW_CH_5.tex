\documentclass[11pt, letterpaper]{article}
\usepackage[includehead, margin=0.75in]{geometry}
\usepackage{fancyhdr}
\usepackage[british]{babel}
\usepackage{sansmathfonts}
\usepackage{soul}
\usepackage{color}
\usepackage{colortbl}
\usepackage{lewis}
\usepackage{siunitx}
\usepackage{textgreek}
\usepackage{mhchem}
\usepackage{modiagram}
\usepackage{tikzorbital}
\usepackage{chemfig}
\usepackage{xspace}
\definecolor{error}{rgb}{255,255,0}
\newcommand{\degree}{\ensuremath{{}^{\circ}}\xspace}
\renewcommand{\familydefault}{\sfdefault}

\begin{document}

\fancypagestyle{plain}{
\fancyhf{}
\fancyhead[L]{CHEM 112 - M01 \\ Dr. Moulder}
\fancyhead[R]{H. Ryott Glayzer \\ \today}
\fancyhead[C]{Assignment \\ \textit{Chapter Name}}
}


\title{Assignment Name}
\author{H. Ryott Glayzer}
\date{\today}


\maketitle


\section*{Notice of ADA Accommodation and Methods}
I have an ADA accommodation to do my assignment on paper.
This document is a utilization of that accommodation.
This assignment will utilize questions from the textbook,
\textit{Chemistry: Atoms First, 2e}, to practice the skills
and learning objectives for this class.

\section{Valence Bond Theory}

\subsection*{Q.1: Explain how $\sigma$ and $\pi$ bonds are similar and how they are different.}


\subsection*{Q.5: A friend tells you that \ce{N2} has three $\pi$ bonds due to the
overlap of the three \textit{p}-orbitals in each Nitrogen atom. Do you agree?}


\section{Hybrid Atomic Orbitals}

\subsection*{Q.9: Explain why a Carbon atom cannot form five bonds using $sp^{3}d$
hybrid orbitals.}


\subsection*{Q.13: Sulfuric acid is manufactured by the series of reactions:\\
\ce{S8(s) + 8O2(g) -> 8SO2(g)}\\
\ce{2SO2(g) + O2(g) -> 2SO3(g)}\\
\ce{SO3(g) + H2O(l) -> H2SO4(l)}\\
Draw the Lewis Structure, predict the molecular geometry by VSEPR,
and determine the hybridization of sulfur for the following:}

\subsubsection*{a. Circular \ce{S8} model}

\subsubsection*{b. \ce{SO2} molecule}

\subsubsection*{c. \ce{SO3} molecule}

\subsubsection*{d. \ce{H2SO4} molecule}


\subsection*{Q.17: Strike-anywhere matches contain a layer of \ce{KClO3} and a
layer of \ce{P4S3}.
The heat produced by the friction of striking the match causes these two compounds
to react vigorously, which sets fire to the wooden stem of the match.
\ce{KClO3} contains the \ce{ClO3-} ion.
\ce{P4S3} is an unusual molecule with the following skeletal structure:}

\chemfig{P?[a]*6(-P(-[:90]S?[b])-P?[a]-[,2]S-P?[b]-S-[,2])}

\subsubsection*{a. Write the Lewis structures for \ce{P4S3} and the \ce{ClO3-} ion.}

\subsubsection*{b. Describe the geometry about the P atoms, the S atom, and the Cl atom in these species}

\subsubsection*{c. Assign a hybridization to the P atoms, the S atom, and tho Cl atom in these species}

\subsubsection*{d. Determine the oxidization states and formal charge of the atoms in \ce{P4S3}
and the \ce{ClO3-} ion.}

\section{Multiple Bonds}

\subsection*{Q.21: The bond energy of a C-C  single bond averages 347 kJ mol^{-1};
that of a \chemfig{C~C} triple bond averages 839 kJ mol^{-1}.
Explain why the triple bond is not three times as strong as the single bond.}


\subsection*{Q.25: Identify the hybridization of the central atom in each of the following
molecules and ions that contain multiple bonds:}

\subsubsection*{a. \ce{ClNO}}

\subsubsection*{b. \ce{CS2}}

\subsubsection*{c. \ce{Cl2CO}}

\subsubsection*{d. \ce{Cl2SO}}

\subsubsection*{e. \ce{SO2F2}}

\subsubsection*{f. \ce{XeO2F2}}

\subsubsection*{g. \ce{ClOF2+}}


\subsection*{Q.29: Draw the orbital diagram for carbon in \ce{CO2} showing how many carbon atom
electrons ore in each orbital.}


\section{Molecular Orbital Theory}

\subsection*{Q.33: Can a molecule with an odd number of electrons ever be diamagnetic?
Explain why or why not.}


\subsection*{Q.37: Explain why an electron in the bonding molecular orbital in the \ce{H2}
molecule has a lower energy than an electron in the 1\textit{s} atomic orbital hydrogen atoms.}


\\
\;
\\

\hline

\\
\;
\\

\section*{Grading}

\begin{center}
	\begin{tabular}{|c|c||c|}
		\hline
		\textbf{Points Possible} & \textbf{Points Earned} & \textbf{Score} \\
		\hline
		32 & X & X/32 \\
		\hline
	\end{tabular}
\end{center}





\end{document}





%%%%%%%%%%%%%%%%%%%%%%%%%%%%%%%%%%%%%%%%%%%%%%%%%%%%%%%%%%%%%%%%%%%%%%%%%%%%%%%
%%%                                 Copypasta                               %%%
%%%%%%%%%%%%%%%%%%%%%%%%%%%%%%%%%%%%%%%%%%%%%%%%%%%%%%%%%%%%%%%%%%%%%%%%%%%%%%%

% Table for questions with multiple parts

% \begin{center}
% 	\begin{tabular}{|c|c|c|c|c|}
% 		\hline
% 		_ & _ & _ & _ & _ \\
% 		\hline
% 		_ & _ & _ & _ & _ \\
% 		\hline
% 	\end{tabular}
% \end{center}
