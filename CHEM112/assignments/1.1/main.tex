\documentclass[11pt, letterpaper]{article}
\usepackage[includehead, margin=0.75in]{geometry}
\usepackage{fancyhdr}
\usepackage[british]{babel}
\usepackage{sansmathfonts}
\usepackage{soul}
\usepackage{siunitx}
%\usepackage{textgreek}
\usepackage{mhchem}
\usepackage{modiagram}
\usepackage{tikzorbital}
\usepackage{chemfig}
\usepackage{xspace}
\newcommand{\degree}{\ensuremath{{}^{\circ}}\xspace}
\renewcommand{\familydefault}{\sfdefault}

\begin{document}

\fancypagestyle{plain}
{
\fancyhf{}
\fancyhead[L]{CHEM 112 - M03 \\ Dr. Moulder}
\fancyhead[R]{H. Ryott Glayzer \\ \today}
\fancyhead[C]{Assignment \\ 1.1}
}


\title{Chemistry Assignment \\ \large Module 1.1}
\author{H. Ryott Glayzer}
\date{\today}


\maketitle


\section*{Notice of ADA Accomodation}
I have an ADA accommodation to do my assignment on paper rather than the online system.
This document is a utilization of that accomodation.
I am having a hard time writing with a pen due to pain, so I am typing and dictating a \LaTeX document.
I will need to speak to Dr. Moulder more in depth about this accomodation this week as it is seeming to be necessary.

\section*{Homework Questions}
\begin{enumerate}
	\item \ce{Al^2+ + C2H3O2 -> Al(C2H3O2)3}
	\item \ce{Hg2^2+ + CO3^2- -> Hg2CO3}
	\item \ce{Al + O -> Al2O3}
	\item \ce{Ca + Br -> CaBr2}
	\item \ce{K + Br -> KBr}
	\item \ce{Sr + Cl -> SrCl2}
	\item \ce{Ca^2+ + ClO^- -> Ca(ClO)2}
	\item \ce{Ca^2+ + PO4^3- -> Ca3(PO4)2}
	\item Chromium(II) Nitrite \ce{-> Cr3N2}
	\item Titanium(IV) Chloride \ce{-> TiCl4}
	\item Calcium Oxide: \ce{Ca^2+ + O^2- -> CaO}
	\item Ferrous Fe(II) and Ferric Fe(III) ions differ by the number of electrons they share (2 vs 3).
	\item Barium Oxide \ce{BaO} is the correct name-formula combo.
	\item Copper(I) Sulfide \ce{CuS} is incorrect. The correct formula is \ce{Cu2S}.
	\item \ce{Fe(NO3)3 * 6H2O} is called Iron(III) Nitrate Hexahydrate
	\item \ce{ClF3} is called Chlorine triflouride
	\item \ce{S2F8} is called disulfur octafluoride.
	\item Cobalt(II) fluoride tetrahydrate is written \ce{CoF2 * 4H2O}.
	\item Diphosphorus pontabramide is written \ce{P2Br5}.
	\item Sulfur tetraiodide is written \ce{SI4}.
	\item Balance \ce{C3H6(g) + O2(g) -> CO2(g) + H2O(g)}.
		\begin{itemize}
			\item \ce{2C3H6(g) + 9O2(g) -> 6CO2(g) + 6H2O(g)}
		\end{itemize}
	\item Balance \ce{PbCO3(s) -> PbO(s) + CO2(g)}.
		\begin{itemize}
			\item \ce{PbCO3(s) -> PbO(s) + CO2(g)} is balanced.
		\end{itemize}
	\item A bar of soap is an example of a mixture.
	\item An iron nail is an example of a mixture. It is likely a mixture of Iron(III) Oxide, elemental iron, and some other elemental metals (making in an alloy of sorts).
	\item Pure water is an example of a compound.
	\item Tearing a piece of paper is an example of a physical change.
	\item Temperature is considered an intensive property. However I have questions (see email sent on 28 Jan 2024).
	\item A chemical change occurs when a diamond is heated to 800 \degree C in an air atmosphere and forms \ce{CO} and \ce{CO2}.
	\item A chemical property of Neon is that it is inert.
	\item The law of multiple proportions explains the constant ration of carbon te hydregon in acetylene gas.
	\item The cathode ray tube experiment determined the existence of electrons.
	\item $^{20}$Ne and $^{22}$Ne are both isotopes of Neon.
	\item Protons and Neutrons are found in the nuclei of atoms.
	\item There are 143 neutrons in an atem of $^{235}$U.
	\item There are 38 protons in an atom of $^{90}$Sr
	\item $^{100}$Ru has 56 neutrons.
	\item $^{60}$Ni contains 28 protons and 32 neutrons.
	\item $^{133}$Cs contains 55 protons, 78 neutrons, and 55 electrons.4
	\item The average atomic mass of element X with isotopes $^{25}$X (80.50\%, 25.03 amu), 
		and $^{27}$X (19.50\%, 26.98 amu) is 25.41 amu.
	\item \ce{SiF4} contains the metalloid Silicon.
	\item \ce{Cu} is the elemental symbol for copper.
	\item Lead has the atomic symbol \ce{Pb}.
	\item Germanium is the third period of group 4A.
	\item Sulfur is chemically similar te Selenium.
	\item Beryllium is an alkaline earth metal.
	\item Helium is a noble gas.
	\item Lithim Chloride (\ce{LiCl}) contains a metal.
	\item Sodium does not have the chemical symbol \ce{S}. It has the chemical symbol \ce{Na}.
\end{enumerate}































\end{document}

