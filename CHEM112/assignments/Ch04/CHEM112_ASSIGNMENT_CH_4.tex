\documentclass[11pt, letterpaper]{article}
\usepackage[includehead, margin=0.75in]{geometry}
\usepackage{fancyhdr}
\usepackage[british]{babel}
\usepackage{sansmathfonts}
\usepackage{soul}
\usepackage{color}
\usepackage{colortbl}
\usepackage{siunitx}
\usepackage{textgreek}
\usepackage{mhchem}
\usepackage{modiagram}
\usepackage{tikzorbital}
\usepackage{chemfig}
\usepackage{xspace}
\usepackage{lewis}
\definecolor{error}{rgb}{255,255,0}
\newcommand{\degree}{\ensuremath{{}^{\circ}}\xspace}
\renewcommand{\familydefault}{\sfdefault}
\usepackage{hyperref}
\hypersetup{
    colorlinks=true,      
    urlcolor=blue
    }
\begin{document}

\fancypagestyle{plain}{
\fancyhf{}
\fancyhead[L]{CHEM 112 - M01 \\ Dr. Moulder}
\fancyhead[R]{H. Ryott Glayzer \\ \today}
\fancyhead[C]{Homework \\ \textit{Ch. 4: Chemical Bonding and Molecular Geometry}}
}


\title{Chapter 4 Homework}
\author{H. Ryott Glayzer}
\date{\today}


\maketitle

\section*{Notice of ADA Accommodation and Methods}
I have an ADA accommodation to do my assignment on paper.
This document is a utilization of that accommodation.
In the \textit{Chemistry: Atoms First, 2e} book,
some homework questions have corresponding answers in the back.
I am doing about half of those questions.

\section{Ionic Bonding}

\subsection*{Q.1: Does a cation gain protons to form a positive charge or does it lose electrons?}
A cation loses electrons to form a positive charge.

\subsection*{Q.5: Predict the charge on the monatomic ions formed
from the following atoms in binary ionic compounds:}

\begin{center}
	\begin{tabular}{|c|c|c|c|c|c|}
		\hline
		P & Mg & Al & O & Cl & Cs \\
		\hline
		3- & 2+ & 3+ & 2- & 1- & 1+ \\
		\hline		
	\end{tabular}
\end{center}


\subsection*{Q.9: Write out the electron configuration for each of the following atoms}

\begin{center}
	\begin{tabular}{|c|c|c|c|c|c|}
		\hline
		Al & Br & Sr & Li & As & S \\
		\hline
		\[ [Ne]3s^{2}3p^{1} \] & \[ [Ar]4s^{2}3d^{10}4p^{5} \] & \[ [Kr]5s^{2} \] & \[ [He]2s^{1} \] & \[ [Ar]4s^{2}3d^{10}4p^{3} \] & \hl{$[Ne]3s^{2}3p^{3}$} \\
		\hline		
	\end{tabular}
\end{center}


\section{Covalent Bonding}

\subsection*{Q.13: Predict which of the following compounds are ionic and which are covalent,
based on the location of their constituent atoms in the periodic table:}

\begin{center}
	\begin{tabular}{|c|c|c|c|c|c|c|c|c|c|c|}
		\hline
		\ce{Cl2CO} & \ce{MnO} & \ce{NCl3} & \ce{CoBr2} & \ce{K2S} & \ce{CO} & \ce{CaF2} & \ce{HI} & \ce{CaO} & \ce{IBr} & \ce{CO2} \\
		\hline
		Covalent & Ionic & Covalent & Ionic & Ionic & Covalent & Ionic & Covalent & Ionic & Covalent & Covalent \\
		\hline		
	\end{tabular}
\end{center}

\subsection*{Q.17: From their positions in the periodic table, arrange the atoms in each of the
	following series in order of increasing electronegativity:}

\begin{center}
	\begin{tabular}{|c|c|c|c|c|}
		\hline
		C, F, H, N, O & Br, Cl, F, H, I & F, H, O, P, S & Al, H, Na, O, P & Ba, H, N, O, As \\
		\hline
		H, C, N, O, F & H, I, Br, Cl, F & P, H, S, O, F & Na, Al, P, H, O & Ba, As, H, N, O \\
		\hline		
	\end{tabular}
\end{center}

\subsection*{Q.21: Identify the more polar bond in each of the following pairs of bonds:}


\begin{center}
	\begin{tabular}{|c|c|c|c|c|c|c|}
		\hline
		HF or HCl & NO or CO & SH or OH & PCl or SCl & CH or NH & SO or PO & CN or NN \\
		\hline
		HF & CO & \hl{SH} & PCl & NH & PO & CN \\
		\hline		
	\end{tabular}
\end{center}

\section{Chemical Nomenclature}

\subsection*{Q.25: Write the formulas of the following compounds:}

\begin{center}
	\begin{tabular}{|c|c|c|c|}
		\hline
		Rubidium Bromide & Magnesium Selenide & Sodium Oxide & Calcium Chloride \\
		\hline
		\ce{RbBr} & \ce{MgSe} & \ce{Na2O} & \ce{CaCl2} \\
		\hline \hline
 		Hydrogen Fluoride &  Gallium Phosphide & Aluminum Bromide & Ammonium Sulfate \\
		\hline
		\ce{HF} & \ce{GaP} & \ce{AlBr3} & \ce{(NH4)2SO4} \\
		\hline
	\end{tabular}
\end{center}

\subsection*{Q.29: Each of the following compounds contains a metal that can exhibit more than
	one ionic charge. Name these compounds:}

\begin{center}
	\begin{tabular}{|c|c|c|}
		\hline
		\ce{Cr2O3} & \ce{FeCl2} & \ce{CrO3} \\
		\hline
		Chromium(III) Oxide & Iron(II) Chloride & Chromium(VI) Oxide \\
		\hline	\hline
		\ce{TiCl4} & \ce{CoCl2*6H2O} & \ce{MoS2} \\
		\hline
		Titanium(IV) Chloride & Cobalt(II) Chloride Hexahydrate & \hl{Molybdenum(II) Sulfide} \\
		\hline
	\end{tabular}
\end{center}

\section{Lewis Symbols and Structures}

\subsection*{Q.34: Write the Lewis Symbols for each of the following ions:}
% Seven Ions

\begin{center}
	\begin{tabular}{|c|c|c|c|c|c|c|}
		\hline
		\ce{As^3-} & \ce{I-} & \ce{Be^2+} & \ce{O^2-} & \ce{Ga^3+} & \ce{Li+} & \ce{N^3-} \\
		\hline
		\lewis{As}{.}{.}{.}{.}{.}{.}{.}{.}$^{3-}$ & \lewis{I}{.}{.}{.}{.}{.}{.}{.}{.}$^{-}$ & 
		\lewis{Be}{}{}{}{}{}{}{}{}$^{2+}$ & \lewis{O}{.}{.}{.}{.}{.}{.}{.}{.}$^{2-}$ & 
		\lewis{Ga}{}{}{}{}{}{}{}{}$^{3+}$ & \lewis{Li}{}{}{}{}{}{}{}{}$^{+}$ & 
		\lewis{N}{.}{.}{.}{.}{.}{.}{.}{.}$^{3-}$ \\
		\hline		
	\end{tabular}
\end{center}

\subsection*{Q.38: Write the Lewis Structure for the diatomic molecule \ce{P2}, an unstable form
of Phosphorus found in high-temperature phosphorus vapor:}
% One Molecule
% \setchemfig{atom style={scale=.85},arrow coeff=.85,atom sep=2.5em}
\begin{center}
	\chemfig{\lewis{\textbf{P}}{\textbf{.}}{\textbf{.}}{}{}{}{}{}{}~\lewis{\textbf{P}}{}{}{}{}{\textbf{.}}{\textbf{.}}{}{}}
\end{center}

\subsection*{Q.42: Write the Lewis Structure for the following:}
% Four Molecules

\begin{center}
	\begin{tabular}{|c|c|c|c|}
		\hline
		\ce{SeF6} & \ce{XeF4} & \ce{SeCl3+} & \ce{Cl2BBCl2} \\
		\hline
		\chemleft[
		\chemfig{Se(-[6]\lewis{F}{.}{.}{}{}{.}{.}{.}{.})(<[4.7]\lewis{F}{.}{.}{.}{.}{}{}{.}{.})(<:[3.3]\lewis{F}{.}{.}{.}{.}{}{}{.}{.})(-[2]\lewis{F}{.}{.}{.}{.}{.}{.}{}{})(<:[0.7]\lewis{F}{}{}{.}{.}{.}{.}{.}{.})(<[-0.7]\lewis{F}{}{}{.}{.}{.}{.}{.}{.})}
		\chemright] & 
		\chemleft[
		\chemfig{Xe(-[6]..)(-[2]..)(<[4.7]\lewis{F}{.}{.}{.}{.}{}{}{.}{.})(<:[3.3]\lewis{F}{.}{.}{.}{.}{}{}{.}{.})(<:[0.7]\lewis{F}{}{}{.}{.}{.}{.}{.}{.})(<[-0.7]\lewis{F}{}{}{.}{.}{.}{.}{.}{.})}
		\chemright]& 
		\chemleft[
		\chemfig{Se(<[5.5]\lewis{Cl}{.}{.}{}{}{.}{.}{.}{.})(<:[4]\lewis{Cl}{.}{.}{.}{.}{}{}{.}{.})(-[2]..)(-[-1]\lewis{Cl}{}{}{.}{.}{.}{.}{.}{.})}
		\chemright]^{+}&
		\chemleft[
		\chemfig{B(-[4.7]\lewis{Cl}{.}{.}{}{}{.}{.}{.}{.})(-[3.3]\lewis{Cl}{.}{.}{.}{.}{.}{.}{}{})-B(-[0.7]\lewis{Cl}{}{}{.}{.}{.}{.}{.}{.})(-[-0.7]\lewis{Cl}{}{}{.}{.}{.}{.}{.}{.})}
		\chemright]\\
		\hline		
	\end{tabular}
\end{center}

\subsection*{Q.46: Methanol, \ce{H3COH}, is used as the fuel in some race cars.
Ethanol, \ce{C2H5OH}, is used extensively as motor fuel in Brazil.
Both methanol and ethanol produce \ce{CO2} and \ce{H2O} when they burn.
Write the chemical equations for these combustion reactions using Lewis Structures
instead of chemical formulas.}

\begin{center}
	\ce{
		2\chemfig{H>:[-0.5]C(-[2]H)(<[5]H)-[7.5]O(<:[5.3]..)(<[6.7]..)-[:30]H} +
		3\chemfig{O=O} ->
		2\chemfig{O=[0.5]C=[-0.5]O} +
		4\chemfig{H-[0.5]O-[-0.5]H}
	}\\
	\ce{
		\chemfig{H-C(-[2]H)(-[6]H)-C(-[2]H)(-[6]H)-O-H} + 3\chemfig{O=O} ->
		3\chemfig{H-[0.5]O-[-0.5]H} + 2\chemfig{O=[0.5]C=[-0.5]O}
	}
\end{center}


% \subsection*{Q.50: The arrangement of atoms in several biologically important molecules is given here.
% Complete the Lewis strictures of these molecules by adding multiple bonds and lone pairs.
% Do not add any more atoms.}
% 5 molecules

\subsection*{Q.54: How are single, double, and triple bonds similar? How do they differ?}
Single, double, and triple bonds are similar in that they are all electron bonds that bind 
atoms together into molecules. 
They differ in their number of electrons involved and the number of $\pi$ and $\sigma$ bonds
that occur.


\section{Formal Charges and Resonance}
\;
\subsection*{Q.58: Sodium Nitrite, which has been used to preserve bacon and other meats,
is an ionic compound. Write the resonance forms of the nitrite ion, \ce{NO2-}.}

\begin{center}
	\begin{tabular}{|c|c|c|c|}
		\hline
		\rowcolor{error}
		\chemfig{O=N=O} & 
		\chemfig{O-N-O} & 
		\chemfig{O~N-O} & 
		\chemfig{O-N~O} \\
		\hline		
	\end{tabular}
\end{center}

\subsection*{Q.62: Determine the formal charge of each element in the following:}
% Four Molecules

\begin{center}
	\begin{tabular}{|c|c|c|c|}
		\hline
		\ce{HCl} & \ce{CF4} & \ce{PCl3} & \ce{PF5} \\
		\hline
		H: 0, Cl: 0 & C: 0, F: 0 & P: 0, Cl: 0 & P: 0, F: 0 \\
		\hline		
	\end{tabular}
\end{center}

% \subsection*{Q.66: Draw all possible resonance structures for each of these:}
% % Four Molecules
% 
% \begin{center}
% 	\begin{tabular}{|c|c|c|c|c|}
% 		\hline
% 		_ & _ & _ & _ & _ \\
% 		\hline
% 		_ & _ & _ & _ & _ \\
% 		\hline		
% 	\end{tabular}
% \end{center}

\subsection*{Q.72: Write the Lewis structures and chemical formula of the compound with the molar
mass of about 70 g/mol that contains 19.7\% nitrogen and 80.3\% Fluorine by mass, and determine
the formal charge of the atoms in this compound.}

\begin{center}
	\chemfig{\lewis{F}{.}{.}{.}{.}{}{}{.}{.}-\lewis{N}{}{}{}{}{}{}{.}{.}(-[2]\lewis{F}{.}{.}{.}{.}{.}{.}{}{})-\lewis{F}{}{}{.}{.}{.}{.}{.}{.}}
\end{center}


\section{Molecular Structure and Polarity}

\subsection*{Q.75: Explain why \ce{HOH} molecules are bent, whereas the \ce{HBeH}
molecule is linear.}

\ce{HOH} is bent because it contains four electron regions, where two of those electron regions
are lone electron pairs.
However, \ce{HBeH} is instead linear because it only has two electron regions,
both of which are single bonds.
This is illustrated below.
\begin{center}
	\chemfig{H-[:36]O(<:[:108]..)(<[:72]..)-[:-36]H}
	\qquad \qquad \qquad
	\chemfig{H-Be-H}
\end{center}

\subsection*{Q.77: Explain the difference between electron-pair geometry and molecular structure.}

Electron-Pair geometry describes where the electron-pair orbitals/regions are in relation to
each atom, while the molecular structure describes the location of the atoms in relation to
each other.

\subsection*{Q.83: What are the electron-pair geometry and molecular structure of each of the
following molecules or ions?}
% 6 Molecules/Ions


\begin{center}
	\begin{tabular}{|c||c|c|c|}
		\hline
		_ & \ce{ClO2-} & \ce{TeCl4^{2-}} & \ce{PH2-} \\
		\hline \hline
		$e^{-}$ Pair Geometry & Tetrahedral & Octahedral & Tetrahedral \\
		\hline		
		Molecular Structure & Bent & Square Planar & Bent \\
		\hline
	\end{tabular}
\end{center}

\subsection*{Q.87: Which of the following molecules and ions contain polar bonds?
Which of these molecules and ions have dipole moments?}
% 7 Molecules/Ions

\begin{center}
	\begin{tabular}{|c||c|c|c|}
		\hline
		_ & \ce{ClO2-} & \ce{TeCl4^{2-}} & \ce{PH2-} \\
		\hline \hline
		Bond Polarity & Polar Covalent & Polar Covalent & Polar Covalent \\
		\hline		
		Dipole Moment? & \hl{No} & No & \hl{No} \\
		\hline
	\end{tabular}
\end{center}


\subsection*{Q.91: The molecule \ce{XF3} has a dipole moment.
Is \ce{X} Boron or Phosphorus?}
\hl{\ce{X} is Boron, since $|\ce{B} - \ce{F}| > |\ce{P} - \ce{F}|$}


\subsection*{Q.95: Describe the molecular structure around the indicated atom or atoms:}
% 9 atoms

\begin{center}
	\begin{tabular}{|c|c|c|c|}
		\hline
		\ce{S} in \ce{(HO)2SO2} & \ce{Cl} in \ce{HOClO2} & \ce{O} in \ce{HOOH} & \ce{N} in \ce{HONO2} \\
		\hline
		\hl{Seesaw} & Trigonal Pyramidal & Bent & \hl{Trigonal Pyramidal} \\
		\hline
	\end{tabular}
\end{center}

\subsection*{Q.99: Draw the Lewis electron dot structures for these molecules, including
resonance structures where appropriate:}

\subsubsection*{a. \ce{CS3^2-}}

\begin{center}
	\chemleft[
	\chemfig{\lewis{S}{}{}{.}{.}{}{}{.}{.}=C(-[6]\lewis{S}{.}{.}{}{}{.}{.}{.}{.})-\lewis{S}{}{}{.}{.}{.}{.}{.}{.}}
	\chemright]^{2-}
	$\longleftrightarrow$
	\chemleft[
	\chemfig{\lewis{S}{.}{.}{.}{.}{}{}{.}{.}-C(=[6]\lewis{S}{.}{.}{}{}{.}{.}{}{})-\lewis{S}{}{}{.}{.}{.}{.}{.}{.}}
	\chemright]^{2-}
	$\longleftrightarrow$
	\chemleft[
	\chemfig{\lewis{S}{.}{.}{.}{.}{}{}{.}{.}-C(-[6]\lewis{S}{.}{.}{}{}{.}{.}{.}{.})=\lewis{S}{}{}{.}{.}{}{}{.}{.}}
	\chemright]^{2-}
\end{center}

\subsubsection*{b. \ce{CS2}}


\begin{center}
	\chemleft[
	\chemfig{\lewis{S}{}{}{.}{.}{}{}{.}{.}=C=\lewis{S}{}{}{.}{.}{}{}{.}{.}}
	\chemright]
\end{center}

\subsubsection*{c. \ce{CS}}

\begin{center}
	\chemleft[
	\chemfig{\lewis{C}{.}{.}{}{}{}{}{}{}~\lewis{S}{}{}{}{}{.}{.}{}{}}
	\chemright]
\end{center}

\subsubsection*{d. Predict the molecular structures for \ce{CS3^2-} and \ce{CS2}. 
Explain how you arrived at this hypothesis.}

I predict that \ce{CS3^2-} is trigonal planar as it has three electron regions and no
lone pairs, while \ce{CS2} is Linear as it has two electron regions and no lone pairs.


% \subsection*{Q.103: Use the simulation at \url{http://openstax.org/1/16MolecPolarity}
% to perform the following exercises for a real molecule.
% You may need to rotate the molecules in three dimensions to see certain dipoles.}
% Three questions
% \textit{This webpage is down.}
% \subsubsection*{a. Sketch the bond dipoles and molecular dipole (if any) for \ce{O3}.
% Explain your observations.}
% 
% \subsubsection*{b. Look at the bond dipoles for \ce{NH3}.
% Use these dipoles to predict whether \ce{N} or \ce{H} is more electronegative.}
% 
% \subsubsection*{c. Predict whether there should be a molecular dipole for \ce{NH3} and,
% if so, in which direction it will point.
% Check the molecular dipole box to test your hypothesis.}
\\
\\

\hline

\\

\section*{Grading and Score}
\begin{center}
	\begin{tabular}{|c|c|c|c|}
		\hline
		\textbf{Points Possible} & \textbf{Points Earned} & \textbf{Score} & \textbf{Percentage} \\
		\hline
		94 & 86 & 86/94 & 91\% \\
		\hline		
	\end{tabular}
\end{center}




\end{document}





%%%%%%%%%%%%%%%%%%%%%%%%%%%%%%%%%%%%%%%%%%%%%%%%%%%%%%%%%%%%%%%%%%%%%%%%%%%%%%%
%%%                                 Copypasta                               %%%
%%%%%%%%%%%%%%%%%%%%%%%%%%%%%%%%%%%%%%%%%%%%%%%%%%%%%%%%%%%%%%%%%%%%%%%%%%%%%%%

% Table for questions with multiple parts

% \begin{center}
% 	\begin{tabular}{|c|c|c|c|c|}
% 		\hline
% 		_ & _ & _ & _ & _ \\
% 		\hline
% 		_ & _ & _ & _ & _ \\
% 		\hline		
% 	\end{tabular}
% \end{center}



