\documentclass[11pt, letterpaper]{article}
\usepackage[includehead, margin=0.75in]{geometry}
\usepackage{fancyhdr}
\usepackage[british]{babel}
\usepackage{sansmathfonts}
\usepackage{soul}
\usepackage{siunitx}
\usepackage{textgreek}
\usepackage{mhchem}
\usepackage{modiagram}
\usepackage{tikzorbital}
\usepackage{chemfig}
\usepackage{xspace}
\usepackage{lewis}
\newcommand{\degree}{\ensuremath{{}^{\circ}}\xspace}
\renewcommand{\familydefault}{\sfdefault}
\usepackage{hyperref}
\hypersetup{
    colorlinks=true,      
    urlcolor=blue
    }
\begin{document}

\fancypagestyle{plain}{
\fancyhf{}
\fancyhead[L]{CHEM 112 - M01 \\ Dr. Moulder}
\fancyhead[R]{H. Ryott Glayzer \\ \today}
\fancyhead[C]{Homework \\ \textit{Ch. 4: Chemical Bonding and Molecular Geometry}}
}


\title{Chapter 4 Homework}
\author{H. Ryott Glayzer}
\date{\today}


\maketitle

\section*{Notice of ADA Accommodation and Methods}
I have an ADA accommodation to do my assignment on paper.
This document is a utilization of that accommodation.
In the \textit{Chemistry: Atoms First, 2e} book,
some homework questions have corresponding answers in the back.
I am doing half of those questions.

\section{Ionic Bonding}

\subsection*{Q.1: Does a cation gain protons to form a positive charge or does it lose electrons?}
A cation loses electrons to form a positive charge.

\subsection*{Q.5: Predict the charge on the monatomic ions formed
from the following atoms in binary ionic compounds:}

\begin{center}
	\begin{tabular}{|c|c|c|c|c|c|}
		\hline
		P & Mg & Al & O & Cl & Cs \\
		\hline
		3- & 2+ & 3+ & 2- & 1- & 1+ \\
		\hline		
	\end{tabular}
\end{center}


\subsection*{Q.9: Write out the electron configuration for each of the following atoms
and for the monatomic ion found in binary ionic compounds for each element:}

\begin{center}
	\begin{tabular}{|c|c|c|c|c|c|}
		\hline
		Al & Br & Sr & Li & As & S \\
		\hline
		\[ [Ne]3s^{2}3p^{1} \] & \[ [Ar]4s^{2}3d^{10}4p^{5} \] & \[ [Kr]5s^{2} \] & \[ [He]2s^{1} \] & \[ [Ar]4s^{2}3d^{10}4p^{3} \] & \[ [Ne]3s^{2}3p^{3} \] \\
		\hline		
	\end{tabular}
\end{center}


\section{Covalent Bonding}

\subsection*{Q.13: Predict which of the following compounds are ionic and which are covalent,
based on the location of their constituent atoms in the periodic table:}

\begin{center}
	\begin{tabular}{|c|c|c|c|c|c|c|c|c|c|c|}
		\hline
		\ce{Cl2CO} & \ce{MnO} & \ce{NCl3} & \ce{CoBr2} & \ce{K2S} & \ce{CO} & \ce{CaF2} & \ce{HI} & \ce{CaO} & \ce{IBr} & \ce{CO2} \\
		\hline
		Covalent & Ionic & Covalent & Ionic & Ionic & Covalent & Ionic & Covalent & Ionic & Ionic & Covalent \\
		\hline		
	\end{tabular}
\end{center}

\subsection*{Q.17: From their positions in the periodic table, arrange the atoms in each of the
	following series in order of increasing electronegativity:}

\begin{center}
	\begin{tabular}{|c|c|c|c|c|}
		\hline
		C, F, H, N, O & Br, Cl, F, H, I & F, H, O, P, S & Al, H, Na, O, P & Ba, H, N, O, As \\
		\hline
		H, C, N, O, F & H, I, Br, Cl, F & P, H, S, O, F & Na, Al, P, H, O & Ba, As, H, N, O \\
		\hline		
	\end{tabular}
\end{center}

\subsection*{Q.21: Identify the more polar bond in each of the following pairs of bonds:}


\begin{center}
	\begin{tabular}{|c|c|c|c|c|c|c|}
		\hline
		HF or HCl & NO or CO & SH or OH & PCl or SCl & CH or NH & SO or PO & CN or NN \\
		\hline
		HF & CO & SH & PCl & NH & PO & CN \\
		\hline		
	\end{tabular}
\end{center}

\section{Chemical Nomenclature}

\subsection*{Q.25: Write the formulas of the following compounds:}

\begin{center}
	\begin{tabular}{|c|c|c|c|}
		\hline
		Rubidium Bromide & Magnesium Selenide & Sodium Oxide & Calcium Chloride \\
		\hline
		\ce{RbBr} & \ce{MgSe} & \ce{NaO} & \ce{CaCl2} \\
		\hline \hline
 		Hydrogen Fluoride &  Gallium Phosphide & Aluminum Bromide & Ammonium Sulfate \\
		\hline
		\ce{HF} & \ce{GaP} & \ce{AlBr3} & \ce{(NH4)2SO4} \\
		\hline
	\end{tabular}
\end{center}

\subsection*{Q.29: Each of the following compounds contains a metal that can exhibit more than
	one ionic charge. Name these compounds:}

\begin{center}
	\begin{tabular}{|c|c|c|}
		\hline
		\ce{Cr2O3} & \ce{FeCl2} & \ce{CrO3} \\
		\hline
		Chromium(III) Oxide & Iron(II) Chloride & Chromium(VI) Trioxide \\
		\hline	\hline
		\ce{TiCl4} & \ce{CoCl2*6H2O} & \ce{MoS2} \\
		\hline
		Titanium(IV) Chloride & Cobalt(II) Chloride Hexahydrate & Molybdenum(II) Sulfide \\
		\hline
	\end{tabular}
\end{center}

\section{Lewis Symbols and Structures}

\subsection*{Q.34: Write the Lewis Symbols for each of the following ions:}
% Seven Ions

\begin{center}
	\begin{tabular}{|c|c|c|c|c|c|c|}
		\hline
		\ce{As^3-} & \ce{I-} & \ce{Be^2-} & \ce{O^2-} & \ce{Ga^3+} & \ce{Li+} & \ce{N^3-} \\
		\hline
		\lewis{As}{.}{.}{.}{.}{.}{.}{.}{.} & \lewis{I}{.}{.}{.}{.}{.}{.}{.}{.} & 
		\lewis{Be}{.}{}{}{}{.}{}{}{} & \lewis{O}{.}{.}{.}{.}{.}{.}{.}{.} & 
		\lewis{Ga}{.}{.}{.}{.}{.}{.}{.}{.} & \lewis{Li}{.}{}{}{}{.}{}{}{} & 
		\lewis{N}{.}{.}{.}{.}{.}{.}{.}{.} \\
		\hline		
	\end{tabular}
\end{center}

\subsection*{Q.38: Write the Lewis Structure for the diatomic molecule \ce{P2}, an unstable form
of Phosphorus found in high-temperature phosphorus vapor:}
% One Molecule
% \setchemfig{atom style={scale=.85},arrow coeff=.85,atom sep=2.5em}
\begin{center}
	\chemfig{\lewis{\textbf{P}}{\textbf{.}}{\textbf{.}}{}{}{}{}{}{}~\lewis{\textbf{P}}{}{}{}{}{\textbf{.}}{\textbf{.}}{}{}}
\end{center}

\subsection*{Q.42: Write the Lewis Structure for the following:}
% Four Molecules

\begin{center}
	\begin{tabular}{|c|c|c|c|}
		\hline
		\ce{SeF6} & \ce{XeF4} & \ce{SeCl3+} & \ce{Cl2BBCl2} \\
		\hline
		\chemleft[
		\chemfig{Se(-[6]\lewis{F}{.}{.}{}{}{.}{.}{.}{.})(<[4.7]\lewis{F}{.}{.}{.}{.}{}{}{.}{.})(<:[3.3]\lewis{F}{.}{.}{.}{.}{}{}{.}{.})(-[2]\lewis{F}{.}{.}{.}{.}{.}{.}{}{})(<:[0.7]\lewis{F}{}{}{.}{.}{.}{.}{.}{.})(<[-0.7]\lewis{F}{}{}{.}{.}{.}{.}{.}{.})}
		\chemright] & 
		\chemleft[
		\chemfig{Xe(-[6]..)(-[2]..)(<[4.7]\lewis{F}{.}{.}{.}{.}{}{}{.}{.})(<:[3.3]\lewis{F}{.}{.}{.}{.}{}{}{.}{.})(<:[0.7]\lewis{F}{}{}{.}{.}{.}{.}{.}{.})(<[-0.7]\lewis{F}{}{}{.}{.}{.}{.}{.}{.})}
		\chemright]& 
		\chemleft[
		\chemfig{Se(<[5.5]\lewis{Cl}{.}{.}{}{}{.}{.}{.}{.})(<:[4]\lewis{Cl}{.}{.}{.}{.}{}{}{.}{.})(-[2]..)(-[-1]\lewis{Cl}{}{}{.}{.}{.}{.}{.}{.})}
		\chemright]^{+}&
		\chemleft[
		\chemfig{Br(-[4.7]\lewis{Cl}{.}{.}{}{}{.}{.}{.}{.})(-[3.3]\lewis{Cl}{.}{.}{.}{.}{.}{.}{}{})-Br(-[0.7]\lewis{Cl}{}{}{.}{.}{.}{.}{.}{.})(-[-0.7]\lewis{Cl}{}{}{.}{.}{.}{.}{.}{.})}
		\chemright]\\
		\hline		
	\end{tabular}
\end{center}

\subsection*{Q.46: Methanol, \ce{H3COH}, is used as the fuel in some race cars.
Ethanol, \ce{C2H5OH}, is used extensively as motor fuel in Brazil.
Both methanol and ethanol produce \ce{CO2} and \ce{H2O} when they burn.
Write the chemical equations for these combustion reactions using Lewis Structures
instead of chemical formulas.}

\begin{center}
	\ce{
		2\chemfig{H>:[-0.5]C(-[2]H)(<[5]H)-[7.5]O(<:[5.3]..)(<[6.7]..)-[:30]H} +
		3\chemfig{O=O} ->
		2\chemfig{O=[0.5]C=[-0.5]O} +
		4\chemfig{H-[0.5]O-[-0.5]H}
	}\\
	\ce{
		\chemfig{H-C(-[2]H)(-[6]H)-C(-[2]H)(-[6]H)-O-H} + 3\chemfig{O=O} ->
		3\chemfig{H-[0.5]O-[-0.5]H} + 2\chemfig{O=[0.5]C=[-0.5]O}
	}
\end{center}


\subsection*{Q.50: The arrangement of atoms in several biologically important molecules is given here.
Complete the Lewis strictures of these molecules by adding multiple bonds and lone pairs.
Do not add any more atoms.}
% 5 molecules

\subsection*{Q.54: How are single, double, and triple bonds similar? How do they differ?}



\section{Formal Charges and Resonance}

\subsection*{Q.58: Sodium Nitrite, which has been used to preserve bacon and other meats,
is an ionic compound. Write the resonance forms of the nitrite ion, \ce{NO2-}.}

\subsection*{Q.62: Determine the formal charge of each element in the following:}
% Four Molecules

\subsection*{Q.66: Draw all possible resonance structures for each of these:}
% Four Molecules

\subsection*{Q.72: Write the Lewis structures and chemical formula of the compound with the molar
mass of about 70 g/mol that contains 19.7\% nitrogen and 80.3\% Fluorine by mass, and determine
the formal charge of the atoms in this compound.}


\section{Molecular Structure and Polarity}

\subsection*{Q.75: Explain why \ce{HOH} molecules are bent, whereas the \ce{HBeH}
molecule is linear.}

\subsection*{Q.79: Explain the difference between electron-pair geometry and molecular structure.}

\subsection*{Q.83: What are the electron-pair geometry and molecular structure of each of the
following molecules or ions?}
% 6 Molecules/Ions

\subsection*{Q.87: Which of the following molecules and ions contain polar bonds?
Which of these molecules and ions hove dipole moments?}
% 7 Molecules/Ions

\subsection*{Q.91: The molecule \ce{XF3} has a dipole moment.
Is \ce{X} Boron or Phosphorus?}

\subsection*{Q.95: Describe the molecular structure around the indicated atom or atoms:}
% 9 atoms

\subsection*{Q.99: Draw the Lewis electron dot structures for these molecules, including
resonance structures where appropriate:}

\subsection*{Q.103: Use the simulation at \url{http://openstax.org/1/16MolecPolarity}
to perform the following exercises for a real molecule.
You may need to rotate the molecules in three dimensions to see certain dipoles.}
% Three questions

\subsubsection*{a. Sketch the bond dipoles and molecular dipole (if any) for \ce{O3}.
Explain your observations.}

\subsubsection*{b. Look at the bond dipoles for \ce{NH3}.
Use these dipoles to predict whether \ce{N} or \ce{H} is more electronegative.}

\subsubsection*{c. Predict whether there should be a molecular dipole for \ce{NH3} and,
if so, in which direction it will point.
Check the molecular dipole box to test your hypothesis.}


\end{document}





%%%%%%%%%%%%%%%%%%%%%%%%%%%%%%%%%%%%%%%%%%%%%%%%%%%%%%%%%%%%%%%%%%%%%%%%%%%%%%%
%%%                                 Copypasta                               %%%
%%%%%%%%%%%%%%%%%%%%%%%%%%%%%%%%%%%%%%%%%%%%%%%%%%%%%%%%%%%%%%%%%%%%%%%%%%%%%%%

% Table for questions with multiple parts

% \begin{center}
% 	\begin{tabular}{|c|c|c|c|c|}
% 		\hline
% 		_ & _ & _ & _ & _ \\
% 		\hline
% 		_ & _ & _ & _ & _ \\
% 		\hline		
% 	\end{tabular}
% \end{center}



