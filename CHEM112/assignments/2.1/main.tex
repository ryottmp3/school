\documentclass[11pt, letterpaper]{article}
\usepackage[includehead, margin=0.75in]{geometry}
\usepackage{fancyhdr}
\usepackage[british]{babel}
\usepackage{sansmathfonts}
\usepackage{soul}
\usepackage{color}
\usepackage{colortbl}
\usepackage{lewis}
\usepackage{siunitx}
\usepackage{textgreek}
\usepackage{mhchem}
\usepackage{modiagram}
\usepackage{tikzorbital}
\usepackage{chemfig}
\usepackage{xspace}
\definecolor{error}{rgb}{255,255,0}
\newcommand{\degree}{\ensuremath{{}^{\circ}}\xspace}
\renewcommand{\familydefault}{\sfdefault}

\begin{document}

\fancypagestyle{plain}{
\fancyhf{}
\fancyhead[L]{CHEM 112 - M01 \\ Dr. Moulder}
\fancyhead[R]{H. Ryott Glayzer \\ \today}
\fancyhead[C]{Assignment \\ \textit{Module 2.1}}
}


\title{Module 2.1 Homework}
\author{H. Ryott Glayzer}
\date{\today}


\maketitle


\section*{Notice of ADA Accommodation and Methods}
I have an ADA accommodation to do my assignment on paper.
This document is a utilization of that accommodation.
This assignment will utilize questions from the textbook,
\textit{Chemistry: Atoms First, 2e}, to practice the skills
and learning objectives for this class.


\section*{Learning Objectives for Module 2.1}


Recognize the different types of chemical transformations: acid-base, precipitation, combination,
decomposition, single-replacement, oxidation-reduction, double replacement, and combustion.
(Chapter 4,5)


\begin{enumerate}
	\item Describe the basic properties of solutions (11.1)
	\item Distinguish electrolyte and nonelectrolyte solutions (chapter 11.1, 11.2)
	\item Identify common acids and bases (chapter 7.2)
	\item Derive chemical equations from narrative descriptions of chemical reactions.(chapter 7.1)
	\item Write and balance chemical equations in molecular, total ionic, and net ionic formats. (7.1)
	\item Define three common types of chemical reactions (precipitation, acid-base, and oxidation reduction) (chapter 7.2)
	\item Classify chemical reactions as one of these three types given appropriate descriptions or chemical equations (7.2)
	\item Predict the solubility of common inorganic compounds by using solubility rules. (chapter 7.2)
	\item Define oxidation, reduction, oxidizing agents, reducing agents, and oxidation numbers (chapter 7.2)
	\item Balance equations for oxidation-reduction reactions in acidic or basic solutions (chapter 7.2)
	\item Describe the reaction with oxygen of organic compounds, metals, and nonmetals (?)
	\item Explain the activity series of metals and use it to predict the product of a redox reactions involving a metal
\end{enumerate}



\section*{11.1: The Dissolution Process}

\subsection*{Q.1: How do solutions differ from compounds? From other mixtures?}

\subsection*{Q.5: }

\section*{11.2: Electrolytes}


\section*{7.1: Writing and Balancing Chemical Reactions}


\section*{7.2: Classifying Chemical Reactions}
























\end{document}





%%%%%%%%%%%%%%%%%%%%%%%%%%%%%%%%%%%%%%%%%%%%%%%%%%%%%%%%%%%%%%%%%%%%%%%%%%%%%%%
%%%                                 Copypasta                               %%%
%%%%%%%%%%%%%%%%%%%%%%%%%%%%%%%%%%%%%%%%%%%%%%%%%%%%%%%%%%%%%%%%%%%%%%%%%%%%%%%

% Table for questions with multiple parts

% \begin{center}
% 	\begin{tabular}{|c|c|c|c|c|}
% 		\hline
% 		_ & _ & _ & _ & _ \\
% 		\hline
% 		_ & _ & _ & _ & _ \\
% 		\hline
% 	\end{tabular}
% \end{center}
