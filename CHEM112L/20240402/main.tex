\documentclass[11pt, letterpaper]{article}
\usepackage[includehead, margin=0.75in]{geometry}
\usepackage{fancyhdr}
\usepackage[british]{babel}


\begin{document}

\fancypagestyle{plain}{%
\fancyhf{}
\fancyhead[L]{CHEM 112L - M60 \\ Dr. Moulder \\TA: Xiaoyu Zhang}
\fancyhead[R]{H. Ryott Glayzer \\\:\\ \today}
\fancyhead[C]{Lab \#11 Pre-Lab Report \\ Gas Laws \\ \:}
}


\title{Lab \#11 Pre-Lab Report \\ \large Gas Laws}
\author{H. Ryott Glayzer}
\date{\today}


\maketitle


\section*{Notice of ADA Accommodation}
I have an ADA accommodation to type out my pre-lab report when my disability flares up.
This document is a utilization of that accommodation.

\section{Question One}
\begin{quote}
    What is the point of this lab? Define the chemical principles we are testing in your own words.
\end{quote}

This lab explores the minor gas laws and the ways that they work together to
contribute to the ideal and combined gas laws.
In a closed system an ideal gas, which is a gas whose molecules take up negligible space and
don't interact with each other, the product of the pressure and volume of a
gas are equal to the product of the number of gas molecules, the temperature
of the gas and the universal gas constant.
No gases are perfectly ideal, but most gases can be modeled to an acceptable level
of significance using the ideal gas law.
The ideal gas law was derived from a combination of several minor gas laws,
namely Boyle's, Charles's and Avagadro's laws.
These laws define the relationship between the volume and temperature of a gas
(Charles's Law), the volume and pressure of a gas (Boyles's Law), and the 
volume and number of molecules of a gas (Avagadro's Law).
Together these laws define the ways that gases interact and vary in relation to
different environmental conditions.

\section{Question Two}
\begin{quote}
    What is the logic of this lab? How do the procedures test the hypothesis that the chemical 
    processes are correct?
\end{quote}

This lab provides direct observations of Boyles's Law, Charles's Law, and the Combined Law.
When testing Boyles's Law, students observe changes in volume while varying the temperature 
of a closed system.
Namely, students will use warm and cold water baths to vary the temperature of a closed-off
syringe and use the graduations on the syringe to measure changes in volume.
While following these procedures, students will observe changes in volume that will validate
the hypothesis that Boyles's Law is correct.
Next, students will test Charles's Law by observing changes in temperature while varying
the volume of a closed system.
This will be achieved using a pressure sensor to measure the changes in pressure while
varying the volume of a closed-off syringe.
These procedures provide direct observations that affirm the validity of Charles's Law.
Finally, students will combine these tests to observe the ways that changes in temperature 
affect both volume and pressure.
Students will use the pressure sensor attached via a tube to measure pressure changes and
the graduations on the syringe to measure volume changes while varying temperature using
warm and cold water baths.
The observations from this experiment will validate the Combined Gas Law.
However, the procedures fail to describe what gas will be used in this experiment and fail
to describe what the nature of the ``pressure sensor'' is. 
These discrepancies make it difficult for students to fully understand the procedures before
performing them.

\section{Question Three}
\begin{quote}
    Where are potential problem points in the procedures? Where is it easy to make an error 
    or have something just go wrong?
\end{quote}

Other than the aforementioned discrepancies in the procedures, this lab provides otherwise
clear and concise descriptions of the experimental procedures.
If students follow the procedures, the instructions of the TA, and general lab guidelines,
there isn't much room for problem points in the procedures.
Accuracy and precision of measurements, as well as consistency in taking the lab seriously,
are going to be the main sources of error and uncertainty in this lab.

\section{Question Four}
\begin{quote}
    What are the health and safety hazards for this lab and how do we minimize them?
\end{quote}

The only perceivable health and safety hazard for this lab is the temperature of the 
warm water bath. If students create a warm water bath with too high a temperature,
it can become a hazard for burns. 
Otherwise, this lab is straight forward and does not present any safety hazards.
As always, following lab procedures, general lab safety guidelines, and TA instructions
are pivotal to ensuring safety in this lab.

\end{document}

