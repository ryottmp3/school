\documentclass[11pt, letterpaper]{article}
\usepackage[includehead, margin=0.75in]{geometry}
\usepackage{fancyhdr}
\usepackage[british]{babel}


\begin{document}

\fancypagestyle{plain}{
\fancyhf{}
\fancyhead[L]{CHEM 112L - M60 / M55 \\ Dr. Moulder \\TA: Xiaoyu / Izabelle}
\fancyhead[R]{H. Ryott Glayzer \\\:\\ \today}
\fancyhead[C]{Lab \#7 Pre-Lab Report \\ Redox Reactions \\ \:}
}


\title{Lab \#7 Pre-Lab Report \\ \large Redox Reactions}
\author{H. Ryott Glayzer}
\date{\today}


\maketitle


\section*{Notice of ADA Accomodation}
I have an ADA accommodation to type out my pre-lab report when my disability flares up.
This document is a utilization of that accomodation.

\section{Question One}
\begin{quote}
    What is the point of this lab? Define the chemical principles we are testing in your own words.
\end{quote}
This lab teaches students the fundamentals of reduction-oxidation, or \textit{redox}, reactions.
We are testing the oxidation of different metals in redox reactions.
This will help students explore chemical reactions and the way they work.
Observations gained in this lab will help students better understand the concepts
that are being explored in lecture.





\section{Question Two}
\begin{quote}
    What is the logic of this lab? How do the procedures test the hypothesis that the chemical 
    processes are correct?
\end{quote}

The logic of this lab is to provide students with hands-on experience in understanding 
and performing redox reactions.
The lab involves mixing different chemicals, observing their behavior under different conditions, 
and recording the results.
The goal is for students to gain a deeper understanding of 
the mechanisms behind redox reactions and how they can be used to perform various tasks.




\section{Question Three}
\begin{quote}
    Where are potential problem points in the procedures? Where is it easy to make an error 
    or have something just go wrong?
\end{quote}

There are several potential problem points in this lab that can lead to errors or problems. Some of 
these include:
\begin{enumerate}
    \item Incorrect mixing of chemicals: If the chemicals are not mixed properly,
        it can lead to incorrect results and potentially dangerous conditions.
    \item  Inadequate safety precautions: Failure to follow proper safety procedures,
        such as wearing protective gear and using appropriate ventilation,
        can lead to accidents and injuries.
    \item Incorrect interpretation of data: If the data collected during the experiment
        is not properly analyzed or interpreted,
        it can lead to incorrect conclusions and decisions.
    \item Lack of proper documentation: Failure to document the experiment and its results
        properly can make it difficult to reproduce the experiment or communicate
        the results effectively.
    \item Incorrect use of chemicals: Using chemicals in an incorrect manner or at the
        wrong time can lead to adverse effects on the environment and human health.
    \item Inadequate cleanup: Failure to properly clean up after the experiment
        can lead to contamination of the laboratory and potential health risks.
    \item Lack of proper communication: Failure to communicate effectively with
        students, instructors, or other parties involved in the experiment
        can lead to misunderstandings and errors.
\end{enumerate}





\section{Question Four}
\begin{quote}
    What are the health and safety hazards for this lab and how do we minimize them?
\end{quote}

If any metals are rather reactive, it may be a health hazard with fire or hydrogen gas.
Any failure to follow instructions from the TA could result in injury to students.
If chemicals are not cleaned up correctly, they could be an unknown chemical hazard to 
any others that are around them.
All in all, these hazards can be avoided by not meting the conditionals provided in the
earlier conditional statements.
As always, clear and direct communication is paramount to success, health, and safety 
in this lab.






\end{document}

