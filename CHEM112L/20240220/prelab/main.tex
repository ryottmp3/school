\documentclass[11pt, letterpaper]{article}
\usepackage[includehead, margin=0.75in]{geometry}
\usepackage{fancyhdr}
\usepackage[british]{babel}


\begin{document}

\fancypagestyle{plain}{
\fancyhf{}
\fancyhead[L]{CHEM 112L - M60 \\ Dr. Moulder \\TA: Xiaoyu Zhang}
\fancyhead[R]{H. Ryott Glayzer \\\:\\ \today}
\fancyhead[C]{Lab \#6 Pre-Lab Report \\ Solubility \\ \:}
}


\title{Lab \#6 Pre-Lab Report \\ \large Solubility}
\author{H. Ryott Glayzer}
\date{\today}


\maketitle


\section*{Notice of ADA Accomodation}
I have an ADA accommodation to type out my pre-lab report when my disability flares up.
This document is a utilization of that accomodation.

\section{Question One}
\begin{quote}
    What is the point of this lab? Define the chemical principles we are testing in your own words.
\end{quote}

This lab explores the solubility of chemicals and substances,
as well as the formation of electrolytes.
Dissolved substances can sometimes undergo chemical reactions
and form precipitates. 
This lab encourages students to explore the formation of precipitates
through direct experimentation.

\section{Question Two}
\begin{quote}
    What is the logic of this lab? How do the procedures test the hypothesis that the chemical 
    processes are correct?
\end{quote}

This lab allows students to explore chemical principles through direct experimentation.
Thus, students will be able to observe solubility, electrolyte formation, and precipitate formation.
This will allow students to develop more understanding of these chemical processes
than through reading alone.

\section{Question Three}
\begin{quote}
    Where are potential problem points in the procedures? Where is it easy to make an error 
    or have something just go wrong?
\end{quote}

In this experiment, errors can occur in several places.
One such error can occur when dropping reactants onto the laminated data table.
Too many drops can cause reactants to mix, which can cause contamination.
Another such error can occur when measuring the formation of electrolytes.
It is mainly a safety issue, where the electricity could shock students.
Likely, the voltage and amperage will be low and any accidents would just hurt a little bit.

\section{Question Four}
\begin{quote}
    What are the health and safety hazards for this lab and how do we minimize them?
\end{quote}

The main safety concern is the utilization of electricity.
Students should follow all of the directions provided by the GTA.
This will ensure that students are safe.






\end{document}

