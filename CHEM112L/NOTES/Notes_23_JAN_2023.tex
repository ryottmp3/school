\documentclass[10pt, letterpaper]{article}
\usepackage[includehead, margin=0.75in]{geometry}
\usepackage{fancyhdr}
\usepackage[british]{babel}
\usepackage{sansmathfonts}
\usepackage{soul}
\usepackage{siunitx}
\sisetup{per-mode=symbol}
\usepackage{textgreek}
\usepackage{mhchem}
\usepackage{modiagram}
\usepackage{tikzorbital}
\usepackage{chemfig}
\usepackage{xspace}
\newcommand{\degree}{\ensuremath{{}^{\circ}}\xspace}
\renewcommand{\familydefault}{\sfdefault}
\setlength{\headheight}{34.54448pt}

\begin{document}

\fancypagestyle{plain}{
	\fancyhf{}
	\fancyhead[L]{CHEM 112L - M60 \\ Dr. Moulder \\TA: Xiaoyu Zhang}
	\fancyhead[R]{H. Ryott Glayzer \\\:\\ \today}
	\fancyhead[C]{Lab Notes \\ Measurement and Uncertainty \\ \:}
}


\title{Lab Notes \\ \large Measurement and Uncertainty}
\author{H. Ryott Glayzer}
\date{\today}


\maketitle


\section*{Notice of ADA Accomodation}
I have an ADA accommodation to type out my pre-lab report when my disability flares up.
This document is a utilization of that accomodation.




\section{Lecture Notes}

\subsection{SI Units and Conversion}

SI units are used in the sciences to convey information better.

To convert \si{\meter} to \si{\centi\meter}:

$\frac{1.70m}{\frac{1m}{100cm}}=1.70m\times\frac{100cm}{1m}=170cm$



\subsection{Significant Figures}

Significant Figures are used to convey error.

\SI{4539}{\meter} has 4 sigfigs.


\subsubsection{logarithms \& exponents}

$\log_{10}{339} = 2.530$

number of sigfigs in the mantissa is the same as the sigfigs in the input.



\subsection{Procedure}

Measure in the center of the meniscus. 
The graduated cylinder is calibrated to the meniscus.





















\end{document}

