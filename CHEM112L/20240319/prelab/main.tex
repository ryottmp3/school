\documentclass[11pt, letterpaper]{article}
\usepackage[includehead, margin=0.75in]{geometry}
\usepackage{fancyhdr}
\usepackage[british]{babel}


\begin{document}

\fancypagestyle{plain}{
\fancyhf{}
\fancyhead[L]{CHEM 112L - M60 \\ Dr.\@ Moulder \\TA: Xiaoyu Zhang}
\fancyhead[R]{H. Ryott Glayzer \\\:\\ \today}
\fancyhead[C]{Lab \#9 Pre-Lab Report \\ Concentration \\ \:}
}


\title{Lab \#9 Pre-Lab Report \\ \large Concentration}
\author{H. Ryott Glayzer}
\date{\today}


\maketitle


\section*{Notice of ADA Accommodation}
I have an ADA accommodation to type out my pre-lab report when my disability flares up.
This document is a utilization of that accommodation.
This pre-lab will not be up to par with my usual work ethic.
I am barely functioning following my kid sister's death last week.
I'm trying.

\section{Question One}
\begin{quote}
    What is the point of this lab? Define the chemical principles we are testing in your own words.
\end{quote}
We are testing titration and concentration determination in this lab.
By adding a titrant to the solution and measuring the known properties,
we can determine the other solution's concentration using math.
We are also testing the Beer-Lambert Law, which relates the absorbance of a 
solution to its molar absorptivity, concentration, and length of light passing through 
the solution.

\section{Question Two}
\begin{quote}
    What is the logic of this lab? How do the procedures test the hypothesis that the chemical 
    processes are correct?
\end{quote}
This lab encourages students to interact hands-on with chemistry concepts to 
understand them better.
The ability of students to perform lab tasks and chemical computations will
also be tested.
The procedures ensure that chemical processes are explored in a way that
basic concentration concepts are evident.

\section{Question Three}
\begin{quote}
    Where are potential problem points in the procedures? Where is it easy to make an error 
    or have something just go wrong?
\end{quote}
Not understanding titration or absorbance would cause issues with this lab.
It is easy to make an error when doing math or taking measurements if
you aren't careful.

\section{Question Four}
\begin{quote}
    What are the health and safety hazards for this lab and how do we minimize them?
\end{quote}
Handling chemicals and glassware are the only potential safety hazards,
but following basic lab safety rules and the TA's instructions will
mitigate risk.

\end{document}

