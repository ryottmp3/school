\documentclass[11pt, letterpaper]{article}
\usepackage[includehead, margin=0.75in]{geometry}
\usepackage{fancyhdr}
\usepackage{fontspec}
\setmainfont{OpenDyslexic}
\usepackage[british]{babel}


\begin{document}

\fancypagestyle{plain}{
\fancyhf{}
\fancyhead[L]{CHEM 112L - M60 \\ Dr. Moulder \\TA: Xiaoyu Zhang}
\fancyhead[R]{H. Ryott Glayzer \\\:\\ \today}
\fancyhead[C]{Lab \#2 Pre-Lab Report \\ Measurement \& Uncertainty \\ \:}
}
\title{Lab \#2 Pre-Lab Report \\ \large Measurement and Uncertainty}
\author{H. Ryott Glayzer}
\date{\today}

\maketitle

\section*{Notice of ADA Accomodation}
I have an ADA accommodation to type out my pre-lab report when my disability flares up.
This document is a utilization of that accomodation.

\section{Question One}
\begin{quote}
    What is the point of this lab? Define the chemical principles we are testing in your own words.
\end{quote}

This lab is designed to give students hands-on experience in taking measurements in the lab.
It is important in all sciences to be able to cite accurate figures and take precise measurements.
Thus, this lab prepares students not only for the rest of the semester, 
but also for the rest of their lives as scientists and engineers.
Significant figures are necessary to convey accurate and precise information to other scientists
and engineers.
This lab teaches students to utilize significant figures in measurements and trains them to communicate
clearly with figures.

\section{Question Two}
\begin{quote}
    What is the logic of this lab? How do the procedures test the hypothesis that the chemical 
    processes are correct?
\end{quote}

This lab ensures students can take measurements and determine uncertainty.
The procedures test the hypothesis by requiring students to take accurate measurements and utilize
significant figures in their lab reports.
This is often a difficult thing for students, who may find the rules of significant figures 



\section{Question Three}
\begin{quote}
    Where are potential problem points in the procedures? Where is it easy to make an error 
    or have something just go wrong?
\end{quote}

For me specifically, standing up for the height measurement may be a problem point.
In general, the utilization of significant figures will turn out to be a problem point for many
students, as the rules tend not to be intuitive at first.
Students may also have problems with the correct measurement of samples with lab equipment.
All of these potential problem points can be worked through by following instructions and
utilizing references given in the lab prep.





\section{Question Four}
\begin{quote}
    What are the health and safety hazards for this lab and how do we minimize them?
\end{quote}

The main hazard is the handling of the unknown liquid. 
Since the liquid's makeup is unknown, it must be treated as if it is something dangerous,
like HF.
This hazard can be minimized by following all safety procedures and following the instructor's
guidance on handling the unknown liquid.

Another hazard comes into play since this will likely be students' first time handling glassware.
Again, following basic safety guidelines and guidance of the instructor will be key to handling
these hazards.




\end{document}

