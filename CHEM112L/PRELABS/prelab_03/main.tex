\documentclass[11pt, letterpaper]{article}
\usepackage[includehead, margin=0.75in]{geometry}
\usepackage{fancyhdr}
%\usepackage{fontspec}
%\setmainfont{OpenDyslexic}
\usepackage[british]{babel}
\usepackage{marvosym}
\renewcommand{\familydefault}{\sfdefault}

\begin{document}

\fancypagestyle{plain}{
\fancyhf{}
\fancyhead[L]{CHEM 112L - M60 \\ Dr. Moulder \\TA: Xiaoyu Zhang}
\fancyhead[R]{H. Ryott Glayzer \\\:\\ \today}
\fancyhead[C]{Lab \#3 Pre-Lab Report \\ Handling Data: Communicating Precision and Accuracy \\ \:}
}


\title{Lab \#3 Pre-Lab Report \\ \large Handling Data: Communicating Precision and Accuracy}
\author{H. Ryott Glayzer}
\date{\today}


\maketitle


\section*{Notice of ADA Accomodation}
I have an ADA accommodation to type out my pre-lab report when my disability flares up.
This document is a utilization of that accomodation.
\Bicycle

\section{Question One}
\begin{quote}
    What is the point of this lab? Define the chemical principles we are testing in your own words.
\end{quote}
This lab teaches students the importance of communicating statistical error in experiments. 
The handout describes determining statistical errors, whether systemic or random, and communicating them effectively
using statistics and mathematics.
I have used $\chi^{2}$-tests to determine goodness-of-fit of a model as well as p-value tests to determine
statistical significance in my undergraduate research.
This lab also teaches the useful skill of organizing data effectively in a spreadsheet.
I have used this as well in my own research, but more commonly use CSV data interfacing with python.
This lab also further hammers in the difference between accuracy and precision.

\section{Question Two}
\begin{quote}
    What is the logic of this lab? How do the procedures test the hypothesis that the chemical 
    processes are correct?
\end{quote}
This lab does not test chemical processes. 
This lab is another teaching lab that communicates necessary skills to students about lab procedure
and data handling.
The logic of this lab is instructional rather than investigative.

\section{Question Three}
\begin{quote}
    Where are potential problem points in the procedures? Where is it easy to make an error 
    or have something just go wrong?
\end{quote}
Potential problem points arise on two fronts: problems with handling data and 
problems with handling lab equipment and chemicals. 
The second front overlaps with safety issues specified in the next section.
Data handling problems may arise if students are new to using spreadsheet software
or have problems with significant figures. 
Problems with chemicals and lab equipment may arise with students handling fire,
acetone, and other less-than-fun chemicals in an unsafe manner.

\section{Question Four}
\begin{quote}
    What are the health and safety hazards for this lab and how do we minimize them?
\end{quote}
Health and safety hazards in this lab may arise from the unsafe handling of fire, acetone,
hexane, and isopropyl alcohol.
These chemicals are all flammable and may pose a risk if handled incorrectly.
These chemicals should also not be ingested and may cause irritation on contact with skin.
In order to mitigate health and safety risks, it is imperative that all necessary precautions
be enacted and lab procedure is followed precisely.
Instructions from the T.A. must alse be followed.

\end{document}
