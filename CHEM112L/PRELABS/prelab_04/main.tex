\documentclass[11pt, letterpaper]{article}
\usepackage[includehead, margin=0.75in]{geometry}
\usepackage{fancyhdr}
% \usepackage{fontspec}
% \setmainfont{OpenDyslexic}
\usepackage[british]{babel}
\renewcommand{\familydefault}{\sfdefault}

\begin{document}

\fancypagestyle{plain}{
\fancyhf{}
\fancyhead[L]{CHEM 112L - M60 \\ Dr. Moulder \\TA: Xiaoyu Zhang}
\fancyhead[R]{H. Ryott Glayzer \\\:\\ \today}
\fancyhead[C]{Lab \#4 Pre-Lab Report \\ The Electromagnetic Spectrum and Electronic Structure \\ \:}
}


\title{The Electromagnetic Spectrum and Electronic Structure}
\author{H. Ryott Glayzer}
\date{\today}


\maketitle


\section*{Notice of ADA Accomodation}
I have an ADA accommodation to type out my pre-lab report when my disability flares up.
This document is a utilization of that accomodation.

\section{Question One}
\begin{quote}
    What is the point of this lab? Define the chemical principles we are testing in your own words.
\end{quote}

The point of this lab is to explore the electromagnetic spectrum and electronic structure 
through various experiments. The chemical principles being tested involve spectroscopy, 
flame tests, and magnetic susceptibility, which help in understanding the behavior of elements 
and compounds under different electromagnetic conditions.


\section{Question Two}
\begin{quote}
    What is the logic of this lab? How do the procedures test the hypothesis that the chemical 
    processes are correct?
\end{quote}

The logic of this lab is to observe and analyze the behavior of substances under different conditions
to confirm known chemical principles. For instance, using a spectroscope allows
for the observation of emission spectra from various light sources, which
confirms the presence of specific wavelengths emitted by different elements.
Flame tests involve observing the characteristic colors emitted when certain
chemicals are heated, providing insights into the electronic structure of atoms.
Magnetic susceptibility experiments help in understanding the magnetic properties
of materials, which can be related to their electronic configuration.


\section{Question Three}
\begin{quote}
    Where are potential problem points in the procedures? Where is it easy to make an error
    or have something just go wrong?
\end{quote}

Potential problem points in the procedures include misalignment of the spectroscope
slit with the light source, which can lead to inaccurate data collection. In the
flame test, overheating the wire or leaving it in the flame for too long can distort
the observed colors. Contamination of chemical bottles during the flame test could
lead to inaccurate results. Additionally, in the magnetic susceptibility experiment,
disturbances such as leaning against the countertop can affect the balance readings,
leading to errors in measurements.


\section{Question Four}
\begin{quote}
    What are the health and safety hazards for this lab and how do we minimize them?
\end{quote}

Health and safety hazards for this lab include potential exposure to chemicals during
the flame test and handling of metal powders in the magnetic susceptibility experiment.
To minimize these hazards, proper personal protective equipment such as gloves and
goggles should be worn. Care should be taken to avoid direct inhalation of chemical
fumes and to handle metal powders cautiously to prevent spills or exposure. Proper
ventilation should be ensured in the laboratory, and all procedures should be conducted
following established safety protocols to minimize risks to students.


\end{document}

